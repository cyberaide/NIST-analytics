\FILE{section-catalog.tex}

\section{Analytics Service Catalogue}
\label{sec:catalog}

\paragraph*{Motivation.}
Cloud providers offered a considerable set of analytics services to
their customers. There are many analytics services available. A user
needs to be able to quickly obtain an overview of such available
services. This helps identifying further actions in order to evaluate
them and identify if further investigation is justified. The catalouge
contains enough details to locate the service and evaluate if it is
useful. However, it may not provide technical details which are
captured by a service registry instead.

\paragraph*{Access Requirements.}
The catalogue may be public or may be restricted while authorized
entities may access it. As analytics services may evolve over time,
time dependent versioned descriptions of the services must be able to
be included. An organizational entity may manage their own
catalogues. It is desirable to have the catalogues be uniform, so that
they can be combined into a larger catalogue combining entries of
multiple organizations.

\TODO{}{8.2.4.2 and 8.2.4.4 and 8.2.6.2 are all labelled Access
  Requirements. perhaps we should be more specific}

\paragraph*{Federation.}
The offerings are typically limited to a particular vendor. Users can
benefit from a federates service catalogue to search and explore for
needed services by the user. In contrast to a registry a catalogue may
not include all technical details but could in contrast include
services that lack such details and thus can be the basis of an
exploratory process.  A Federated analytics service repository is
planned to be hosted on GitHub (LINK TBD) The catalogue contains the
following attributes, many of which are also used in an analytics
service registry.

The catalogue is organized as a list of entries, where each entry
contains a number of attributes. These attributes may be required or
optional. We list in Table \ref{tab:cat}.


\begin{table}[htb]
\caption{Catalouge attributes}
\label{tab:cat}
\begin{tabular}{p{3cm}p{10cm}p{2cm}}
Name	& Description	& Required \\
\hline
ID	& UUID, globally unique	& \OK \\
Name	& Name of the service	& \OK \\
Title	& Human readable title 	& \OK \\
Public	& True if Public 
(needs use case to delineate what pub private means) & 	\OK \\
Description	& Human readable short Description of the Service	& \OK \\ 
Version	& The version number or tag of the service	& \OK \\
License	& The license description	& \OK \\
Microservice & 	\OK/No/Mixed	& \OK \\
Protocol	& REST	& \OK \\
Owner	& Name of the distributing entity, organization or individual. It could be a vendor.	& \OK \\
Modified	& Modification Timestamp (when first same as created)	\OK \\
Created	& Date on which the entry was first created	& \OK \\
Documentation	& Link to a URL with detailed description of the service	& O \\
Source	& Link to the source code if available	& O \\
Tag(s)	& Human readable common tags that are used to identify the service that are associated with the service	& O \\
Category(s)	& A category that this service belongs to (NLB, Finance, ...)	& O \\ 
Specification/ Schema	& Pointer to where schema is located &	O \\
Additional metadata	& Pointer to where additional is located including the one here.	& O \\
Endpoint	& The endpoint of the service	& O \\
SLA/Cost	&	& O \\
Authors	& contact details of the people or organization responsible for the service (freeform string)	& O \\
Data	& Description on how data is managed	& O \\
\hline
\end{tabular}
\OK = Required; O = Optional
\end{table}
