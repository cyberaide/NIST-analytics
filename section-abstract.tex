

\begin{abstract}

\FILE{section-abstract.tex}

NEW

Over the last several years, the computation landscape for conducting
data analytics has completely changed. While in the past a lot of
the activities have been undertaken in isolation by companies and
research institutions, today's infrastructure constitutes a wealth of
services offered by a variety of providers that offer
opportunities for reuse and interactions.

In this document, we document how to expand analytics services to
focus on developing a framework for reusable hybrid multi-service data
analytics. It includes (a) a short technology review that explicitly
targets the intersection of hybrid multi-provider analytics services
(b) enhancing the concepts of services to showcase how hybrid, as well
as multi-provider services, can be integrated and reused via the
proposed framework, (d) address analytics service composition, and (c)
integrate container technologies to achieve state-of-the-art analytics
service deployment capabilities.
 



OLD

This document summarizes the NIST Analytics Service
Framework that targets to analytics functionalities to be hosted on
computational resources including Clouds, Containers, and High
Performance Computing (HPC). Although we use the REresentational
State Transfer (REST) to formulate some details of the architecture,
it is independent from REST and can be formulated in other
frameworks. While using REST we use a familiar pattern for architect,
implementer, and strategists. Due to the many frameworks, programming
languages and services supporting REST the architecture can easily be
enhanced and implemented with various technical solutions.  The
analytics framework also targets big data. Big data is a term used to
describe extensive datasets, primarily in the characteristics of
volume, variety, velocity, and veracity. While opportunities exist
with Big Data analytics, the data characteristics can overwhelm
traditional technical approaches, and the growth of data is outpacing
scientific and technological advances in data analytics. To advance
progress in Big Data analytics, the NIST Big Data Public Working Group
(NBD-PWG) is working to develop consensus on important fundamental
concepts related to Big Data. The results are reported in the NIST Big
Data Interoperability Framework (NBDIF) series of volumes.

\end{abstract}
