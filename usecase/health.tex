\subsection{Health care}

\TODO{This section does not include how analytics services are used and we use it}

A technician in the hospital uses voice commands to control an MRI
machine to take tomographic images. The images will be automatically
sent to a private analytics service to identify if the images contain
signs of COVID-19. In this case. multiple services are consulted to
assure that the best available and appropriate algorithms are chosen.
Once identified, an image can be sent to a public analytics service
(given patient consent).

\TODO{the following ppg needs more than editing}

In order to improve the available images to improve the deep learning
data analysis. Previous images that have tested negative may be
reanalyzed with the newly improved models. If new cases are found
based on the improved analysis, health care providers are notified,
and further actions regarding the treatment plan by the supervising
physician is cast. As we can see, this example has all the ingredients
that we need to create a new generation of services that integrates
on-premise infrastructure, public and private services. The
orchestration of the services as well as a convenient
interface. Automation of the workflow of this use case example is
explicitly stated.

Analyzing this example, we identify a mixture of services that utilize
on-premise infrastructure (the MRI), private and public services. A
variety of service patterns are used in concert to establish an analysis
pipeline targeted explicitly for this application use case. 

The goal of this work is to analyze the interoperability of such
service scenarios and identify such patterns motivating a vendor
neutral architecture that promotes reusable implementations to 
support aspects of similar use cases addressed by them.


