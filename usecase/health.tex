\subsection{Health care}

\Data Analytics as a service holds great promise for more accurate healthcare diagnostics and even prediction of disease outcomes for patients. 
 
A-Critical Care Diagnoses
Data analytic and ML tools to help prevent overcrowding in the hospital unit. Research has shown that patients coming to the hospital with a particular type of heart attack did not require hospitalization in the ICU. By enhancing the medical informatics analytic as a service, Emergency Room (ER) clinicians identify these patients and refer them to noncritical care. This service would improve the quality of care for patients and reduce unnecessary costs—high Rick stable patients with Non-ST-Segment-Elevation.
In the United States, ≈40% of patients with non–ST-segment–elevation myocardial infarction (NSTEMI) who initially present without cardiogenic shock or cardiac arrest is admitted to the intensive care unit (ICU). ICU utilization for these initially stable NSTEMI patients varies substantially between care providers. 
The severity of illness on presentation is similar for NSTEMI patients treated and not treated in the ICU. Therefore risk-based ICU utilization—admitting those at the highest risk of developing complications requiring ICU care to the ICU and admitting those at lower risk to a non-ICU setting—has the potential to align resource use with patient needs.
For patients with initially stable NSTEMI, the Acute Coronary Treatment and Intervention Outcomes Network (ACTION) ICU risk score uses data available at the time of hospital presentation to predict the risk of clinical deterioration requiring ICU care— in-hospital death, cardiac arrest, shock, respiratory failure, heart block requiring pacemaker placement, and stroke. Risk scores are infrequently used in clinical practice, but embedding a risk score calculator with associated decision support into the electronic 
health records (EHR) could increase uptake. 
 
 B- Radiology
Imaging methods include computed tomography (CT), magnetic resonance imaging (MRI), and X-ray, offering analytic/computational capabilities that process images with incredible speed and accuracy. Recognizes complex patterns to assess a patient's health, even predictions of disease outcomes for patients. However, sophisticated analytic as a service needs more robust evaluation methods to reduce risk to the patient, establish trust, and ensure wider adoption. A clear example is many researchers' difficulty in classifying imaging results from early studies of the coronavirus disease that spawned the COVID-19 pandemic.



TODO{This section does not include how analytics services are used and we use it}

A technician in the hospital uses voice commands to control an MRI
machine to take tomographic images. The images will be automatically
sent to a private analytics service to identify if the images contain
signs of COVID-19. In this case. multiple services are consulted to
assure that the best available and appropriate algorithms are chosen.
Once identified, an image can be sent to a public analytics service
(given patient consent).

\TODO{the following ppg needs more than editing}

In order to improve the available images to improve the deep learning
data analysis. Previous images that have tested negative may be
reanalyzed with the newly improved models. If new cases are found
based on the improved analysis, health care providers are notified,
and further actions regarding the treatment plan by the supervising
physician is cast. As we can see, this example has all the ingredients
that we need to create a new generation of services that integrates
on-premise infrastructure, public and private services. The
orchestration of the services as well as a convenient
interface. Automation of the workflow of this use case example is
explicitly stated.

Analyzing this example, we identify a mixture of services that utilize
on-premise infrastructure (the MRI), private and public services. A
variety of service patterns are used in concert to establish an analysis
pipeline targeted explicitly for this application use case. 

The goal of this work is to analyze the interoperability of such
service scenarios and identify such patterns motivating a vendor
neutral architecture that promotes reusable implementations to 
support aspects of similar use cases addressed by them.


This use case has the implict requirements needing the following
aspects to be addressed by the framework we develop.

\begin{enumerate}

\item{\bf AS vendor neutral cloud and computer service integration.} ...

  \begin{enumerate}
  \item {\bf AS in cloud.} ...
  \item {\bf AS in LCCF.} ...
  \item {\bf AS in microservices.} ...
  \end{enumerate}

\item{\bf AS architecture.} ...

  \begin{enumerate}
  \item{\bf AS vendor neutral interfaces.} ...
  \item{\bf AS REST.} ...
  \item{\bf AS layers such as interface, service layer, and provider layer.} ...
  \end{enumerate}

\item{\bf AS workflow.} ...

  \begin{enumerate}
  \item{\bf AS catalog and registry.} ...
  \item{\bf AS cooperation.} ...
  \item{\bf competition.} ...
  \item{\bf AS orchestrator.} ...
  \end{enumerate}


\item{\bf AS calculation.}

  \begin{enumerate}
  \item{\bf AS with DL.} ...
  \item{\bf AS data analytics.} ...
  \end{enumerate}

\item{\bf AS security.} ...

\end{enumerate}


