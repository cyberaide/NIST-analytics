\FILE{section-concepts.tex}


\section{Enabling Concepts}\label{s:background}

This section explains the motivation while briefly summarizing the
different concepts constituting our work. While our previous work
focused on developing a Big Data Reference Architecture and standards
roadmap \cite{nist-v8}. This work specifically focused on the
definition of {\bf\em Analytics Services}. This work is a logical
enhancement to the earlier work and can leverage activities conducted
as part of the NIST Big Data Reference Architecture (NBD-RA) and
NBD-RA Interfaces. However, the work here targets explicitly {\bf\em
Data Analytics} as a pathway to integrate the data analytics
ecosystem. This includes not only existing legacy analytics services
and tools but also the integration of state-of-the-art AI services,
including machine learning and deep learning analytics, within the
auspice to create a service-oriented framework integrating all of
them. Hence, they can easily be reused by others. Next, we define some
of the terminology and concepts we use.


\begin{description}

\item[From Big Data Reference Architecture to Analytics Services.]
\label{s:arch} NIST has developed a Big Data Reference Architecture as part of
NBDIF\cite{nist-v6} and identified a number of use cases that motivate
it \cite{nist-v3}. We leverage this effort
~\cite{nist-v1,nist-v2,nist-v3,nist-v4,nist-v5,nist-v6,nist-v7,nist-v8,nist-v9}
while formulating service interoperability specifications that we
focus on in this effort and have not been previously addressed in
detail. While we previously focused mostly on infrastructure
management, this effort enhances the activities to include high-level
coordinated service deployments and utilization while leveraging
containers. The concepts we introduce next specifically target
analytics services and not just infrastructure services.  However, the
lessons learned from the earlier work significantly influences this
activity.


\item[Hybrid Analytics Services.]

A {\em hybrid analytics service} combines the strength of analytics
services that are offered by providers in public, private, or
on-premise usage scenarios. It leverages them to provide optimized
orchestration across private, public, and on-premise
analytics. Optimization benefits are not limited to reducing cost but
also addressing security and privacy concerns when the data analytics
or the data to perform the analytics can not be hosted in public
clouds. Many of the major cloud providers such as AWS, Azure, Google,
IBM, Oracle, and others have made hybrid clouds a cornerstone of their
business model, with each of them essentially promoting their own
solutions. Recently, however, we see that the cloud provider's focus
is no longer offering just infrastructure but instead to provide
services hiding and abstracting the cloud infrastructure entirely from
the users while placing focus on offering services. This has lead to
vendors also providing hybrid analytics solutions that may integrate
multiple services offered by various providers, resulting in solutions
with heterogeneous service offerings. Integrating such services
involves significant challenges, as each vendor may conform to and or
require different integration solutions for addressing various public,
private and on-premise analytics services. Customers will generally
benefit from a more integrated approach to ease deployment and
management concerns.

\item[Multi-Analytics Services to Cooperate and Compete.]

Over the last several years, we have seen an explosion of analytics
services, mainly through the integration of AI and deep learning
services. High-level analytics services are being developed that hide
and abstract the complex infrastructure needed to embed not only
services from one vendor but multiple vendors. Hence we speak of {\em
multi-analytics services}. These services can then be used in {\em
cooperation} and/or {\em competition}. We cooperate if services
enhance each other, we compete if a service is chosen over another
service due to better service level agreements. Through this interplay
of the services, it is beneficial to formalize interoperability
between them. In cases of competition, we also need to be able to
formulate a competing service that then calls out other services to
implement desired analytics tasks.

\item[Identification of State-of-the-Art Data Analytics Patterns.]

Analytics services consumers have to ask why and how now,
this opportunity can be addressed to enable this interplay by utilizing
hybrid multi-service based analytics as a service needs are clearly
motivated by state-of-the-art data analytics capabilities that have
only recently become available. In addition, government agencies have
provided some of the most capable high-end computing systems over the
last years, they tightly integrated specialized GPUs as well as
container technologies to bring forward new data analytics
capabilities in these on-premise services. Industry has provided
advanced analysis capabilities for some time but have only recently
reached a maturity supporting reuse and cooperation opportunities
between them.

\FILE{section-fair.tex}

\item[FAIR Principle for Analytics Services.]
\label{sec:fair}

Reusability is an essential part of adaptation. To make it explicitly
clear, we adopt the well-known FAIR principal \cite{??} but enhance
them first by focussing on analytics services, deployability, and
operations. Together we use the tern Analytics Service FAIR Principle
(AS-FAIR-DO).

To project easy reusability, we strive toward the implementation of
the AS-FAIR-DO principle for analytics services. The FAIR principle is
typically applied to data; as such, we can apply it to the metadata
associated with analytics services. The FAIR principle addresses which
to be findable, accessible, interoperable, and reusable. In
Figure \ref{fig:as-fair-do} we explicitly augmented the general FAIR
principle with terminology so it can apply to analytics services. The
augmentations are colored in red. As such, not only data is part of the
principle, but also the data representing the services themselves.

\end{description}

\begin{figure*}[htb]
\centering\resizebox{1.0\columnwidth}{!}{
%\begin{tabular}{p{1cm}p{8cm}}
\begin{tabular}{p{1cm}p{12cm}}
\multicolumn{2}{l}{To be Findable:} \\
F1 & \textcolor{red}{analytics services metadata} are assigned a globally unique and persistent identifier \\ 
F2 & \textcolor{red}{analytics services} data are described with rich metadata (defined by R1) \\
F3 &  \textcolor{red}{analytics services metadata} clearly and explicitly include the identifier of the data related to the analytics services it describes \\ 
F4 & \textcolor{red}{analytics services metadata} are registered or indexed in a searchable resource \\
\multicolumn{2}{l}{To be Accessible:} \\
A.1 &  \textcolor{red}{analytics services metadata} are retrievable by their identifier using a standardized communications protocol \\
    A1.1 & \textcolor{red}{analytics services} the protocol is open, free, and universally implementable \\
    A1.2 & the \textcolor{red}{analytics services} protocol allows for an authentication and authorization procedure, where necessary \\ 
A.2 & \textcolor{red}{metadata} are accessible, even when the data are no longer available \\
\multicolumn{2}{l}{To be Interoperable:}\\
I1. & \textcolor{red}{analytics services metadata} use a formal, accessible, shared, and broadly applicable language for knowledge representation. \\
I2. &  \textcolor{red}{analytics services metadata} use vocabularies that follow FAIR principles \\
I3. &  \textcolor{red}{analytics services metadata} include qualified references to other metadata \\
\multicolumn{2}{l}{To be Reusable:} \\
R1. & \textcolor{red}{analytics services metadata} are richly described with a plurality of accurate and relevant attributes \\
R1.1 & (meta)data are released with a clear and accessible data usage license \\
R1.2 & (meta)data are associated with detailed provenance \\
R1.3 & (meta)data meet domain-relevant community standards \\
\multicolumn{2}{l}{To be Deployable:}\\
D.1 & \textcolor{red}{analytics services metadata describing deployability aspects} \\
\multicolumn{2}{l}{To be Operational:} \\ 
O.1 & \textcolor{red}{analytics services metadata describing operational aspects} \\ 
\end{tabular}
}
\caption{Fair guiding principles adapted to analytics services:
  Analytics Services - FAIR - Deployable and Operational
  (AS-FAIR-DO).}\label{fig:as-fair-do}
\end{figure*}

