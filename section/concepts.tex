\FILE{section-concepts.tex}


\section{Concepts}\label{s:background}

In this section, we explain the motivation while summarizing briefly,
the different concepts constituting our work.

While our previous work focused on developing a Big Data Reference
Architecture and  standards roadmap \cite{nist-v8}. This work
specifically focused on the definition of {\bf\em Analytics Services}.
This work is a logical enhancement to the earlier work and can
leverage activities conducted as part of the NIST Big Data Reference
Architecture (NBD-RA) and NBD-RA Interfaces.  However, the work here
targets explicitly {\bf\em Data Analytics} as a pathway to integrate
the data anlaytics ecosystem. This includes not only existing of
legacy analytics services and tools, but also the integration of
stat-of-the-art AI services including machine learning and deep
learning analytics within the auspice to create a service oriented
framework integrating all of them. Hence, they can easily be reused by
others.

Next we define som of the terminology and concepts we use.


\subsection{From Big Data Reference Architecture to Analytics Services}
\label{s:arch}

NIST has developed a Big Data Reference Architecture as part of
NBDIF\cite{nist-v6} and identified a number of use cases that motivate
it \cite{nist-v3}. We leverage this effort
~\cite{nist-v1,nist-v2,nist-v3,nist-v4,nist-v5,nist-v6,nist-v7,nist-v8,nist-v9}
while formulating service interoperability specifications that we
focus on in this effort and has not been previously addressed in
detail. While we previously focused mostly on infrastructure
management, this effort enhances the activities to include high-level
coordinated service deployments and utilization while leveraging
containers. The concepts that we introduce next are specifically
targeting analytics services and not just infrastructure services.
However the lessons learned from the earlier work significanly
influsences this activity.


\subsection{Hybrid Analytics Services}

A {\em hybrid analytics service} combines the strength of analytics
services that are offered by providers in public, private, or
on-premise usage scenarios. It leverages them in order to provide
optimized orchestration across private public, and on-premise
analytics. Optimization benefits are not limited to reducing cost, but
also to address security and privacy concerns when the data analytics
or the data to perform the analytics can not be hosted in public
clouds. Many of the major cloud providers such as AWS, Azure, Google,
IBM, Oracle, and others have made hybrid clouds a cornerstone of their
business model, with each of them essentially promoting their own
solutions. Recently, however, we see that the cloud provider's focus
is no longer offering just infrastructure but instead to provide
services hiding and abstracting the cloud infrastructure entirely from
the users while placing focus on offering services.  This has lead to
vendors also providing hybrid analytics solutions that may integrate
multiple services offered by various providers, resulting in solutions
with heterogeneous service offerings. The integration of such services
involves significant challenges, as each vendor may conform to and or
require different integration solutions for addressing various public,
private and on-premise analytics services. In general, customers will
benefit from a more integrated approach to ease deployment and
management concerns.

\subsection{Multi-Analytics Services to Cooperate and Compete}

Over the last several years, we have seen an explosion of analytics
services, mainly through the integration of AI and deep learning
services. High-level analytics services are being developed that hide
and abstract the complex infrastructure needed to embed not only
services from one vendor but multiple vendors. Hence we speak of {\em
multi-analytics services}. These services can then be used in {\em
cooperation} and/or {\em competition}. We cooperate if services
enhance each other, we compete if a service is chosen over another
service due to better service level agreements. Through this interplay
of the services, it is beneficial to formalize interoperability
between them. In cases of competition, we also need to be able to
formulate a competing service that then calls out other services to
implement desired analytics tasks.

\subsection{Identification of State-of-the-Art Data Analytics Patterns}

Analytics services consumers have to ask why and how now,
this opportunity can be addressed to enable this interplay utilizing
htbrid multi-service based analytics as a service needs are clearly
motivated by state-of-the-art data analytics capabilities that have
only recently become available.  In addition, government agencies have
provided some of the most capable high-end computing systems over the
last years, they have tightly integrated specialized GPUs as well as
container technologies to bring forward new data analytics
capabilities in these on-premise services. Industry has provided
advanced analysis capabilities for some time but have only recently
reached a maturity supporting reuse and cooperation opportunities
between them.

