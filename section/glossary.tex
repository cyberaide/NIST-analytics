\FILE{section-glossary.tex}

\section{Glossary}
\label{sec:glossary}

\TODO{remove all terms that are not used or are important.}

This Glossary provides terms that are used in this document. In
addition we have provided a definitions in Section \ref{sec:defining}
to focus on the details of some terms and terminology used in this
document specifically focus sing on Analytics Services.

\begin{description}

  
\item[AAI:] Authentication and Authorization
  Infrastructure. Facilitates single, virtualized identities (issued
  by the {\em user's home organization.})

\item[AARC:]   \TODO{Russell: The Authentication and Authorisation for
  Research and Collaboration project. I will write a descriptive
  sentence for it that you can add later. Also, below, ACL: access
  control list? I have a bunch other acronyms with descriptions that I
  can just send you as a list that you can choose from and add if you
  wish. Under Iaas, p is probably referring to PaaS}  

\item[ABAC:] attribute based access control

\item[ACL:] \TODO{Russell}{TBD}

\item[ACID:] Atomicity, Consistency, Isolation, Durability.

\item[Analytics:] The systematic analysis of data, to uncover patterns
  and relationships between data, historical trends, and attempts at
  predictions of future behaviors and events.

\item[Analytics management:] A sub function within the [metadata]
  registry.

\item[Analytics services] azure cognitive, google analytics, aws
  [dozens], watson analytics... in contrast to ML frameworks like
  tensorflow, pytorch, caffe2, and in contrast to Programming
  libraries like python, scikit, shiny, or R Studio [??]


\item[Analytics Workflow:] The sequence of processes or tasks part of
  the analysis

\item[API:] Application Programming Interface

\item[ASCII:] American Standard Code for Information Interchange

\item[BASE:] Basically Available, Soft state, Eventual consistency
  Classification scheme per 11179, a container of the classifiers for
  all kinds of administered items including common data elements
  [CDE]s.

\item[CIA:] Confidentiality, Integrity, and Availability.

\item[CLI:] Command Line Interface.

\item[Consumer:] Ametadata consumer, per IHE, is responsible for the
  import of metadata created by the source. In the context of section
  A.3,

\item[Container:] See
  \url{http://csrc.nist.gov/publications/drafts/800-180/sp800-180_draft.pdf}
  Cloud Computing The practice of using a network of remote servers
  hosted on the Internet to store, manage, and process data, rather
  than a local server or a personal computer. See
  \url{http://nvlpubs.nist.gov/nistpubs/Legacy/SP/nistspecialpublication800-145.pdf}.

\item[CDE:] Common data element = smallest meaningful data container
  in a given context.

\item[DDC:] data dictionary component. library of data elements that
  are used to establish common understanding of the meaning of coding
  systems.

\item[Data element:] Describes [or defines] the logical unit of
  data. Per 11179, the element refers to the structure of the data,
  distinct from a data instance.

\item[Data element concept:] the combination of an object class, and a
  related property.

\item[DEX:] Data element exchange = interoperability profile. Enables
  retrieval of extraction specifications for data elements which are
  defined in particular domains. ``Options'' including the cross
  enterprise doc sharing [XDS] doc type binding option and the cross
  community access [XCA] doc type binding option, extend basic DEX
  functionality, addressing interoperability with Secondary Data
  Usage[s]. Allowing secondary users to know if and where [data] is
  available when it is organized as a doc sharing environment,
  I.e. XDS, MPQ, XCA.

\item[DevOps:] A clipped compound [?] [portmanteau?] of software
  DEVelopment and information technology OPerationS \TODO{improve}

\item[Deployment:] The action of installing software on resources

\item[DMTF:] Distributed Management Task Force. A standards
  organization.

\item[Extraction Specification:] a map of data locations used as a
  guide for extracting data. SPARQL, SQL, and XPath scripts, aka
  mapping scripts, are examples of specifications for locating a data
  element in a particular content model.

\item[FIM:] federated identity management. A core component of AAI.


\item[Federated database system:] two definitions: 1. a system that
  maps multiple autonomous database systems using a combining scheme
  where one DB interface is provided for local / owner access to data,
  and another simpler interface is provided for guest access to non
  owner data. 2. a DBMS which is an element of a federated group, that
    allows members belonging to the same federated group, to access
    data residing in the DBMS.

\item[HTTP:] HyperText Transfer Protocol HTTPS HTTP Secure

\item[Hybrid Cloud:] See
  \url{http://nvlpubs.nist.gov/nistpubs/Legacy/SP/nistspecialpublication800-145.pdf}.


\item[IaaS:] Infrastructure as a Service SaaS Software as a Service
  Implementation.

\item[IGTF:] Interoperable Global Trust Federation.

\item[ITL:] Information Technology Laboratory metadata data employed
  to annotate other data with descriptive information.

\item[IHE:] \TODO{Russel, TBD}

\item[LDAP:] Lightweight Directory Access Protocol. A
  directory/registry standard.

\item[Metadata generator:] A sub function within the repository
  Metadata Registry [MDR] a database that manages the semantics of
  data elements, and this case, provides discovery and analytics
  management services.

\item[MRR:] Metadata registry / repository = specialized DB of
  metadata which describe data constructs.

\item[Microservice:] Architecture Is an approach to build applications
  based on many smaller modular services. Each module supports a
  specific goal and uses a simple, well-defined interface to
  communicate with other sets of services.

\item[NBDIF] \TODO{}{TBD}
  
\item[NBD-PWG:] NIST Big Data Public Working Group.

\item[NBDRA:] NIST Big Data Reference Architecture.

\item[NBDRAI:] NIST Big Data Reference Architecture Interface.

\item[NIEM:] National information exchange model = government wide
  standards based approach to exchanging information in the US.

\item[NIST:] National Institute of Standards and Technology.

\item[OGF:] Open Grid Forum.

\item[OS:] Operating System.

\item[P2P:] Peer to Peer.

\item[PKI:] Public Key Infrastructure. a security related certificate
  aka X.509.

\item[Proxy:] \TODO{}{TBD}

\item[Registry:] \TODO{}{TBD}
  
\item[Registry, federated:] \TODO{}{TBD}


\item[REST:] REpresentational State Transfer Retrieval a transaction
  where a system returns a selection I.e. a list of data elements from
  a database, or in the scope of this document, a list of elements in
  a metadata registry.

\item[SAML:] Security assertion markup language. a security standard;
  web browser service that defines ``syntax and semantics to exchange
  auth and auth data between security domains.'' Not compatible with
  other authentication protocols such as Secure socket?, OIDC, etc.

\item[Serverless Computing:] Serverless computing specifies the
  paradigm of function as a service (FaaS). It is a cloud computing
  code execution model in which a cloud provider manages the function
  deployment and utilization while clients can utilize them. The
  charge model is based on execution of the function rather than the
  cost to manage and host the VM or container.

\item[Services:] \TODO{TBD}

\item[Service registry:] in the context of an SOA architecture, this
  registry is a network based directory that contains available
  services.

\item[Software stack:] A set of programs and services that are
  installed on a resource to support applications.  Value domain the
  description of a permissible set of values for the property of a
  data element definition.

\item[XACML] eXtensible Access Control Markup Language. a security
  related standard developed by OASIS, circa 2005.

\end{description}


