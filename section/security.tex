\FILE{section-security.tex}

\subsection{Security}
\label{sec:security}

Analytics services have a variety of security needs. This includes
authentication, authorization and audit (Section \ref{sec:aaa}). Data
and use may also need to be secured and privacy concerns need to be
addressed (Section \ref{sec:privacy}).

\subsubsection{Authentication, Authorization and Audit (AAA)}\label{sec:aaa}

Authentication and Authorization functions would be needed across the
Analytics Service Framework to authenticate and authorize
human-to-machine (H2M) and machine-to-machine (M2M) communication
between the interface layer and the service layer and between the
service layer and the provider layer.

Communication to the services in the Service Layer would be via
RESTful APIs. For authenticating to services in the Service Layer,
Single-Sign on (SSO) using token-based access control schemes like
OAuth, SAML, or Kerberos will be used. Most Web Service frameworks
support these SSO mechanisms out-of-the-box. Once the client is
authenticated, authorization checks are done to ensure the client is
authorized for requested CRUD operation on the REST resource.

Authentication and authorization for machine-to-machine (M2M) access
between the service layer and Private and Public Cloud in provider
layer will be done using Cloud's Role-based Access Control (RBAC) and
Attribute-based Access Control (ABAC) functions and Identity and
Access Management (IAM) policies. Service accounts are provisioned in
the cloud environment for service clients, and these service accounts
are given appropriate roles based on Principles of least privilege
(PoLP).

Client actions such as access to resources, changes to configuration
and changes to roles and access policies should be continuously
monitored and audited. Most web service frameworks have basic audit
and logging functions which can create audit trails on files,
databases, or remote logging services. Audit on cloud providers can be
done by enabling cloud audit trails and cloud alerts.

\subsubsection{Privacy}\label{sec:privacy}

The Analytics Service Framework will have appropriate controls in
place for ensuring data privacy. The framework will support encryption
of data at rest and in transit, both intra-layer and inter-layer, to
safeguard against unwanted access of data. Any data that is copied
over for processing purposes by the framework will be deleted once the
processing is done. The framework will also allow administrators to
delete inactive users and service accounts and revoke
accesses. Sensitive attributes in the data will be masked from users
and services based on roles and policies. The framework will also
audit and log any access to data which can then be monitored in a
Security Incident Event Management System (SIEM) for unwanted or
abnormal access.

There are some other privacy controls which the broader ecosystem
should have which would be out of scope for the Analytics Service
Framework because they don't belong there. Some of these include
having a data inventory to map data storage, classification of all
data and governance processes in place to ensure data literacy and
manage data lifecycle within an organization.

\subsubsection{Federation}

As some services in the analytics framework may be distributed across
a variety of services owned by different providers it is beneficial to
allow the integration of a service federation. NIST has spearheaded an
extensive document \cite{nist-Lee2020} addressing such federation that
could be leveraged and federated services could be developed based on
is.

\begin{comment}
\subsubsection{Artifacts}

function
data
logs and  audit

\subsubsection{Privacy}

privacy
    input
    output
    function
    
asynchronous events, how does privacy apply
batch functions
streaming functions

data
\end{comment}
