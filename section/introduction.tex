\FILE{section-introduction.tex}

\newcommand{\WG}{\TODO{Wo: define name of wrking group}{WORKINGGROUP}}

\section{Introduction}


\subsection{Background}

\TODO{add related research section here}

Keywords:

* cloud

* rest services

* microservices

* hybrid

* fair

* analytics as a service

* software as a service

* data analytics as a service

* data mining

* computation

* descriptive analytics to describe and summarize information from data

* function analytics

* machine learning integration


\subsection{Scope and Objectives}

NBD-PWG\footnote{\TODO{introduce in the background section. THe
background section has been integrated here. check valitity and fix
somehow.}} is exploring how to extend
NBDIF \footnote{\TODO{introduce in the background section. THe
background section has been integrated here. check valitity and fix
somehow.}} for packaging scalable analytics as services to meet the
challenges of today's information analytics. These services are
intended to be reusable, deployable, and operational for Big Data,
High Performance Computing, AI machine learning (ML), and deep
learning (DL) applications, regardless of the underlying computing
environment.

This document explores key focus areas and document level of interest
from industry, government, and academia in extending the NBDIF to
develop scalable analytics as services that are reusable, deployable,
and operational, regardless of the underlying computing environment.
\TODO{Russel: note that the 'string' reusable, deployable, and
  operational was also used in the previous ppg.}


The work has been conducted with input from the NIST BIg Data Working
group while enhancing their original activities to address
requirements for Analytics Service. This includes hosted on
computational resources including Clouds, Containers, and High
Performance Computing (HPC), thus targeting analytics services hosted
on premis, private and public clouds. We have chosen REpresentational
State Transfer (REST) to formulate some details of the architecture,
it is independent from REST and can be formulated in other
frameworks. While using REST we use a familiar pattern for architect,
implementer, and strategists. Due to the many frameworks, programming
languages and services supporting REST the architecture can easily be
enhanced and implemented with various technical solutions.


TBD:

The
analytics framework also targets big data.
Big data is a term used to
describe extensive datasets, primarily in the characteristics of
volume, variety, velocity, and veracity. While opportunities exist
with Big Data analytics, the data characteristics can overwhelm
traditional technical approaches, and the growth of data is outpacing
scientific and technological advances in data analytics.

To advance
progress in Big Data analytics, the NIST Big Data Public Working Group
(NBD-PWG) is working to develop consensus on important fundamental
concepts related to Big Data. The results are reported in the NIST Big
Data Interoperability Framework (NBDIF) series of volumes.

