\FILE{section-catalog.tex}

\subsubsection{Analytics Service Catalogue}
\label{sec:catalog}

\paragraph*{Motivation.}
Cloud providers offer a considerable set of analytics services to
their customers. There are many analytics services available, and a user
needs to be able to quickly obtain an overview of such available
services. This helps identify further actions to evaluate
them and identify if further investigation is justified. The catalog
contains enough details to locate the service and evaluate its
usefulness. However, it may not provide technical details 
captured by a service registry instead.

\paragraph*{Access Requirements.}
The catalog may be public or may be restricted while authorized
entities may access it. As analytics services may evolve. Hence,
time-dependent versioned descriptions of the services must be able to
be included. An organizational entity may manage its own
catalogs. It is desirable to have the catalogs be uniform so that
they can be combined into a larger catalog combining entries of
multiple organizations.

\paragraph*{Federation.}
The offerings are typically limited to a particular vendor. Users can
benefit from a federated service catalog to search and explore for
needed services by the user. In contrast to a registry, a catalog may
not include all technical details but could, in contrast, include
services that lack such details and thus can be the basis of an
exploratory process. A Federated analytics service repository is
planned to be hosted on GitHub \TODO{(LINK TBD)}. The catalog contains the
following attributes, many of which are also used in an analytics
service registry.

The catalog is organized as a list of entries, where each entry
contains a number of attributes. These attributes may be required or
optional. We list in Table \ref{tab:reg} in the column Catalog.


