\FILE{section-background.tex}


\section{Concepts}\label{s:background}

In this section, we explain the motivation while summarizing briefly,
the different concepts constituting our work.

While our previous work focused on developing a Big Data Reference
Architecture [or standards roadmap] \cite{vol8}. This work
specifically focused on the definition of {\bf\em Analytics Services}.
This work is a logical enhancement to the earlier work and can
leverage activities conducted as part of the NIST Big Data Reference
Architecture (NBD-RA) and NBD-RA Interfaces.  However, the work here
targets explicitly {\bf\em Data Analytics} as a pathway to integrate
the data anlaytics ecosystem. This includes not on;y forward looking
efforts, but also the integration of legacy analytics, as well as
machine learning and deep learning analytics within the auspice to
create a service oriented framework integrating all of them, so they
can easily be reused by others.


\subsection{Hybrid Analytics Services}

A {\em hybrid analytics service} (HAS) combines the strength of analytics
services that are offered by providers in public, private, or
on-premise usage scenarios. It / A \TODO{[HAS]} leverages them \TODO{[x]} in order to provide ideally optimized
orchestration across them \TODO{[x]}. Benefits are not limited to reducing cost,
but also to address security and privacy concerns when the data
analytics or the data to perform the analytics can not be hosted in
public clouds. Many of the major cloud providers such as AWS, Azure,
Google, IBM, Oracle, and others have made hybrid clouds a cornerstone
of their business model, with each of them essentially promoting their own
solutions. Recently, however, we see that the cloud provider's focus is no longer
offering just infrastructure but instead to provide services hiding /abstracting the
cloud infrastructure entirely from the users. 
This has lead to vendors
also providing hybrid analytics solutions that may integrate
multiple services offered by various providers, resulting in solutions
with heterogeneous service offerings. The integration of such services
involves significant challenges, as each vendor may conform to and or require different
integration solutions for addressing various public, private and on-premise
analytics services. In general, customers will benefit from a more
integrated approach to ease deployment and management concerns.

\subsection{Multi-Analytics Services}

Over the last two years, we have seen an explosion of analytics
services, mainly through the integration of AI and deep learning
services. High-level analytics services are being developed that hide
and abstract the complex infrastructure needed to embed not only
services from one vendor but multiple vendors. Hence we speak of {\em
multi-analytics services}. These services can then be used in
cooperation or competition. We cooperate if services enhance each
other, we compete if a service is chosen over another service due to
better service level agreements. Through this interplay of the
services, it is beneficial to formalize interoperability between
them. In cases of competition, we also need to be able to formulate a
competing service that then calls out other services to implement
desired analytics tasks.

\subsection{Enhancing the Big Data Reference Architecture}
\label{s:arch}

NIST has developed a Big Data Reference Architecture as part of
NBDIF\cite{nist-v6} and identified a number of use cases that motivate
it \cite{nist-v3}. We can leverage the effort
~\cite{nist-v1,nist-v2,nist-v3,nist-v4,nist-v5,nist-v6,nist-v7,nist-v8,nist-v9}
while formulating novel interoperability specifications that arise in
our effort that has not been previously addressed. While we previously focused
mostly on virtual machine management, this effort will enhance the
activities to include high-level coordinated service deployments and
utilization while leveraging containers.

\subsection{Identification of State-of-the-Art Data Analytics Patterns}

Concerned analytics services consumers have to ask why and how now,
this opportunity can be addressed to enable this interplay utilizing
htbrid multi-service based analytics as a service needs are clearly
motivated by state-of-the-art data analytics capabilities that have
only recently become available.  while government agencies have
provided some of the most capable high-end computing systems over the
last years, they have tightly integrated specialized GPUs as well as
container technologies to bring forward new data analytics
capabilities in these on-premise services. Industry has provided
advanced analysis capabilities for some time but have only recently
reached a maturity supporting reuse and cooperation opportunities
between them.

With respect to all of the aforementioned challenges, we articulate 
the motivation behind our work. 


