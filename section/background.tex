\FILE{section-background.tex}



\section{Motivation}\label{s:background}

In this section, we explain the motivation  while summarizing
briefly, the different concepts constituting our work.

With Big Data's compound annual growth rate at 61 percent and its
ever-increasing deluge of information in the mainstream, the
collective sum of world data will grow from 33 zettabytes (ZB, 1021)
in 2018 to 175 ZB by 2025 \NOTE{likely outdated by now. find out
current trends and numbers}. The presence of such a rich source of
information requires a massive analysis that can effectively bring
about much insight and knowledge discovery. While our previous work
focused on developing a Big Data Reference
Architecture [or standards roadmap] \cite{??}. This work specifically focused on the
definition of {\bf\em Analytics Services}.  This work is a logical
enhancement to the earlier work and can leverage activities conducted
as part of the NIST Big Data Reference Architecture (NBD-RA) and
NBD-RA Interfaces.  However, the work here targets explicitly {\bf\em
Data Analytics} as a pathway to integrate the data anlaytics ecosystem. This includes not on;y forward looking efforts, but also the integration of legacy analytics,
as well as machine learning and deep learning analytics within the auspice to
create a service oriented framework integrating all of them, so they can easily be reused by others.



\subsection{Hybrid Analytics Services}

A {\em hybrid analytics service} (HAS) combines the strength of analytics
services that are offered by providers in public, private, or
on-premise usage scenarios. It / A \NOTE{[HAS]} leverages them \NOTE{[x]} in order to provide ideally optimized
orchestration across them \NOTE{[x]}. Benefits are not limited to reducing cost,
but also to address security and privacy concerns when the data
analytics or the data to perform the analytics can not be hosted in
public clouds. Many of the major cloud providers such as AWS, Azure,
Google, IBM, Oracle, and others have made hybrid clouds a cornerstone
of their business model, with each of them essentially promoting their own
solutions. Recently, however, we see that the cloud provider's focus is no longer
offering just infrastructure but instead to provide services hiding /abstracting the
cloud infrastructure entirely from the users. 
This has lead to vendors
also providing hybrid analytics solutions that may integrate
multiple services offered by various providers, resulting in solutions
with heterogeneous service offerings. The integration of such services
involves significant challenges, as each vendor may conform to and or require different
integration solutions for addressing various public, private and on-premise
analytics services. In general, customers will benefit from a more
integrated approach to ease deployment and management concerns.

\subsection{Multi-Analytics Services}

Over the last two years, we have seen an explosion of analytics
services, mainly through the integration of AI and deep learning
services. High-level analytics services are being developed that \NOTE{hide / abstract} the
complex infrastructure needed to embed not only services from one
vendor but multiple vendors. Hence we speak of {\em multi-analytics
  services}. These services can then be used in cooperation or
competition. We cooperate if services enhance each other, we compete
if a service is chosen over another service due to better service
level agreements. Through this interplay of the services, it is
beneficial to formalize \NOTE{this / any / all} interoperability. In cases of competition,
we also need to be able to formulate a competing service that then
calls out other services to implement desired analytics tasks.

\subsection{Enhancing the Big Data Reference Architecture}
\label{s:arch}

NIST has developed a Big Data Reference Architecture as part of
NBDIF\cite{nist-v6} and identified a number of use cases that motivate
it \cite{nist-v3}. We can leverage the effort
~\cite{nist-v1,nist-v2,nist-v3,nist-v4,nist-v5,nist-v6,nist-v7,nist-v8,nist-v9}
while formulating novel interoperability specifications that arise in
our effort that has not been previously addressed. While we previously focused
mostly on virtual machine management, this effort will enhance the
activities to include high-level coordinated service deployments and
utilization while leveraging containers.

\subsection{Identification of State-of-the-Art Data Analytics Patterns}

Concerned analytics services consumers have to ask why and how now, 
this opportunity can be addressed to enable this interplay? \NOTE{[? what interplay?]} 
analytics as a service needs are clearly motivated by state-of-the-art data
analytics capabilities that have only recently become available.
while government agencies have provided some of the most capable
high-end computing systems over the last years, they have \NOTE{[tightly?]} integrated
specialized GPUs as well as container technologies to bring forward
new data analytics capabilities in these on-premise services. \NOTE{[commercial]} Industry
has provided advanced analysis capabilities for some time but have only
recently reached a maturity supporting reuse and
cooperation opportunities between them. 

With respect to all of the aforementioned challenges, we articulate 
the motivation behind our work. 

example usecases. 

1. health care

A technician in the hospital uses voice commands to control an MRI
machine to take tomographic images. The images will be automatically
sent to a private analytics service to identify if the images contain
signs of COVID-19. In this case. multiple services are consulted to 
assure that the best \NOTE{available / appropriate} algorithms are chosen. 
Once identified, an image can be sent to a public analytics service (given patient consent).
\NOTE{[hey the following ppg needs more than editing.]}
in
order to improve the available images to improve the deep learning data
analysis. Previous images that have tested negative may be reanalyzed
with the newly improved models. If new cases are found based on
the improved analysis, health care providers are notified, and further
actions regarding the treatment plan by the supervising physician is
cast. As we can see, this example has all the ingredients that we need
to create a new generation of services that integrates on-premise
infrastructure, public and private services. The orchestration of the
services as well as a convenient interface. Automation of the workflow
of this use case example is explicitly stated.

Analyzing this example, we identify a mixture of services that utilize
on-premise infrastructure (the MRI), private and public services. A
variety of service patterns are used in concert to establish an analysis
pipeline targeted explicitly for this application use case. 

The goal of this work is to analyze the interoperability of such
service scenarios and identify such patterns motivating a vendor
neutral architecture that promotes reusable implementations to 
support aspects of similar use cases addressed by them.
