\FILE{section-usecases.tex}


\section{Use Cases for Analytics Services}
\label{sec:usecases}


\TODO{We need s single use case that uses the word analystics services (AS) and showcases:}


\TODO{The usecase should motivate the use of an AS catalog/registry,
AS workflow, AS security, AS vendor neutral interfaces, AS vendor
neutral cloud service integration, AS orchestrator. Motivation for AS
layers, such as interface, service layer, and provider layer.}


\TODO{Reminder: The reason we initially picked WRF and HVAC is that they seem initially
disconnected services, but if you look closer HVAC could integarte WRF to
increase forcast .... so this is example for cooperation and workflow integration).}

HERE STARTS OLD TEXT:

Our work is motivated by a number of uses cases. The use cases were
contributed by community members that were interested in this work.


It includes healthcare, security, numerical weather prediction and
HVAC optimization. The use cases are explained in more detail at:

\begin{itemize}
\item wrf: \url{https://github.com/cyberaide/NIST-analytics/blob/main/usecase/wrf.tex}
\item HVAC: \url{https://github.com/cyberaide/NIST-analytics/blob/main/usecase/hvac.tex}
\item Security: \url{https://github.com/cyberaide/NIST-analytics/blob/main/usecase/security.tex}
\item Health: \url{https://github.com/cyberaide/NIST-analytics/blob/main/usecase/health.tex}
\end{itemize}

Each usecase includes a general explanation about the problem that the
uses case adresses and also explicitly comments on requirements needed
for an alanytics service derived form the individual use case
perspective.

\TODO{Gregor will enable PDF versions of the documents and make them available in github.}

We are explaining form these usecases one of them in order to keep the paper short. 

\TODO{explanation of the use case}


This use case has the implict requirements needing the following
aspects to be addressed by the framework we develop. Other use cases
may not need all of these requirements, but benefit from a framework
that can address a subset of them.

\begin{description}
\item[AS cooperation.]
\item[AS competition.]
\item[AS in cloud.]
\item[AS in LCCF.]
\item[AS vendor neutral cloud and computer service integration.]
\item[AS with DL.]
\item[AS REST.]
\item[AS in microservices.]
\item[AS data analytics.}
\item{AS catalog/registry.]
\item[AS workflow.]
\item[AS security.]
\item[AS vendor neutral interfaces.]
\item[AS orchestrator.]
\item[AS lyers such as interface, service layer, and provider layer.}


\end{description}


