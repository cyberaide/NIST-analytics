\FILE{section-data}
\subsection{Data Management}
\label{sec:data}

As data is to be integrated and analyzed as well as produced the
framework must include a sufficiently broad ability to access data from
a wide variety of sources and utilize many different services.  This
includes files, object stores, and databases.  To address the access of
file-based data we have integrated into cloudmesh components to deal
with transfers and copies from remote resources, as well as developed
a component that deals with many file transfers through Globus. The
later overcomes restrictions due to file numbers and improves transfer
speed, and access speed.

Managing data in analytics services is an important implicit
requirement. The data has to be readily accessible and may have to be
pre-staged to the resources where the computation is performed. Also
one has to make sure that policy restrictions are appropriately dealt
with in order to perform the analytics tasks. The policy restrictions
typically include the total size of the data for a particular user or
group but also could include the number of files.

For this reason, it is advantageous to have a service that can deal with
such restrictions. Unfortunately, such services are not readily
available based on our experience with different HPC centers offering
compute resources for analytics tasks and jobs.

Available services are typically restricted to filesystems that are
accessible on the compute nodes as well as services that copy between
local computers or between compute centers. The later is frequently
covered by `rsync` over SSH or UDP, or through
Globus \cite{www-globus-transfer} as a service. However, when we tried
using Globus we found that it is not usable when millions of files are
involved, but performs well when in such cases a tar file is produced
over many files and the tar file is transfered in a single
operation. We also encountered frequent timeouts on the servers that
were involved in a server-server transfer using many file transfers.

To simplify this we developed a program
cloudmesh-globus \cite{cloudmesh-globus} that allows us to specify an
entire directory with many files that first automatically package the
directory and transfers the compressed file to the remote machine
where it then gets uncompressed and placed in the appropriate file
system. Hence such steps have not to be done by hand by the
researcher, but are done automated providing a simplified
filesystem-to-filesystem service via Globus without issues.

Other alternatives could include
cloudmesh-storage \cite{cloudmesh-storage} which include prototype
transfer services even among cloud providers such as amazon, azure,
and Google, while leveraging a compute services conducting file
copies between the involved parties.


