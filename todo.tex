\FILE{todo.tex}
\section{TODO}

\TODO{}{Describe service provider, library provider infrastructure
  provider... Move this sub section to its own 'concept' section, or
  to some other separate section, perhaps the {\em Landscape}.}

\TODO{}{potential text for describe service provider, infra provider:
  describing federated services and infrastructure providers but
  likely better for a Federated Access Landscape section:}

Two essential parts of a ‘federation’ infrastructure are: the
definition, advertising, and discovery of services; and the secure
access to the services (via federated identity).  Infrastructure
capabilities for ‘unified service discovery’ in federated service
registries are not mature. Service / infrastructure providers
especially, face many challenges. For example, various registry
infrastructures provide support for different user access methods /
technologies, but too often are not compatible with the heterogeneous
access methods of other (xyz) which desire to interoperate with the
federation.  The following technology review briefly outlines existing
projects that have implemented bindings or integrations of service
discovery (SD) and federated identity management and user
authentication technologies, as they may relate to the federated
access.

Sub section 1: 	Notable Implementations and Related Technologies

Sub section 2:	Identity Federation

Sub section 3: 	Terms, Definitions, Acronyms ({gregor} move to appropriate location)

Section 1a: Notable federated access infrastructures 

Type A: 

Infrastructures which provide resources to a broad audience with a
wide range of user requirements. Oriented more toward sharing, and
potentially supportive of ‘open science;’ but also requiring more
attention to AAI.

XSEDE.
The extreme science and engineering discovery environment is an
HPC and grid based information system. Supports a number of identity
capabilities, and credential translation; and supports several
integrations and provides a wide range of resources.

LHC. The Large Hadron Collider Computing Grid.

EOSC-Hub, a super federation of the EGI, EUDAT, and INDIGO projects,
also serving a broad audience of users.

EGI and EUDAT. Both the European Data Infrastructure and the European
Grid Infrastructure are federated environments / collaborative
infrastructures providing functionality for: (AAI requirements (in the
    form of:)) SSO, delegation, non-web federated access, multiple
authentication protocols, LoAs, and distributed authorization; and (SD
  requirements (in the form of:)) service discovery, unified API,
replication, hierarchies, service ingo lifecycle mgmt., and common
information model. Both EGI and EUDAT provide a command line interface
option for user access. EGI is X.509 based.

Type B: Infrastructures which provide resources of a domain specific
nature to smaller communities.

CLARIN (Common Language Resources and Technology Infrastructure)
research infrastructure. A SAML based federation of repositories,
supporting SSO for members, and utilizing a registry discovery service
(driven by the Language Resource Switchboard (a text analysis
    application). Unlike EGI and EUDAT, CLARIN provides a user
  workflow engine. CLARIN does not provide a command line interface as
  a user access method; and does not support multiple authentication
  protocols or LoAs.

Section 1b: Related technologies / middleware(s), and projects: 

AARC Blueprint architecture (BPA) project. Active in harmonization.
Argus. An authorization service based on SAML 2.0, VOMS, and X.509
specifications. No GUI. Limited scalability.

B2ACCESS: a suite of AAI proxy services including identity management,
FIM, and credential translation. Deployed by notable research
infrastructures including CLARIN, EOSC-Hub, and EPOS.

BDII. An information service infrastructure service discovery
technology deployed by EGI. BDII presents several challenges, in part
due to its reliance on LDAP.  Distributed UDDI Deployment Engine
(DUDE) improves on UDDI, a service discovery technology. Limited
scalability.  eduGAIN; and the Research and education identity
federations (REFEDS). eduGAIN is a SAML based federation interconnect
/ federated AAI, integrated with ELIXIR. REFEDS is an ongoing project
to develop attribute harmonization across national federations.

ELIXIR. A large research infrastructure. Provides federated
authentication / AAI technology for web browser style access but does
not support non web style access. Supports low LoA social network
identities. No support for credential translation. Integrates with
eduGAIN.

European Middleware Initiative (EMI) Registry. An open source client –
server federated registry that enables federated authentication,
service discovery, and other cross infrastructure services. EMIR
relies on JSON, GLUE, and X.509; and integrates with B2ACCESS for
AAI. European Plate Observing System (EPOS). Very large research
infrastructure with thousands of users.

FIWARE. An initiative providing an authorizations API, exposing an
HTTP interface, ergo it has no GUI. Does not support delegation of
policies.

GOCDB. An information service infrastructure service registry
technology deployed by EGI and EUDAT.  GLUE (2.0). Grid Lab Uniform
Environment information model. Standards-based / “open standard.”

IGTF. Interop global trust federation. An infrastructure
authentication technology used by CLARIN, EGI, and EUDAT.  Helmholtz
data federation project. Development of an AAI proxy that provides
credential translation and interoperability between different
providers with incompatible protocols, resulting in federated identity
management.

Metacomputing directory service (MDS); and ISIS. Peer to peer service
discovery technologies.

Moonshot. Federated authentication / AAI project, offering ‘non web’
browser access. Developed by IETF. Several shortcomings include lack
of support for OIDC or X.509.

MWSDI. Groups registries into domains, and groups domains into
federations. Limited scalability.  ORCHID. A commercial, author
oriented identity service provider with limited integration capability
and low levels of assurance. SCOPUS is another author oriented
identity service provider.

PRACE. A large European HPC infrastructure based on PKI AAI. Limited
to X.509 services; does not offer OIDC or SAML based identity
federation.

SIRTFI. Security incident response trust framework for federated
identify. Part of REFEDS.

UNICORE. Centralized HPC service registry middleware. The UNICORE FTP
is not accessible through web browser interfaces. ckre CIM / CIS.

VOMS. Virtual organization management service. A user authorization
technology deployed by EGI.

WSO2 Identity Server. An authorization service with Active Directory,
JDBC, and LDAP integration. Not suitable for distributed management of
access control policies. Tightly bound to FIM, which creates
challenges for other systems which have established authentication
schemes.

AT&T XACML. Xacml is an RA. NDG XACML

OpenAZ 

Axiomatics policy server (commercial) The incomplete parts of the
listing can be supplemented by russell.  Research and education
federations group assurance frame (RAF) Section 2: Identity Federation
Architectures Simplistically stated, user identity is the first step
in federated authentication, and connection to target infrastructure /
service provider endpoints. Generally speaking, there are three types
of architectures.

Type 1: ‘Mesh:’ the most widely adopted model (74), however
substantially more complex than the others.

Type 2: Hub and spoke with distributed login: AARC’s Blueprint
Architecture (BPA), and B2ACCESS are adopters of this architecture.

Type 3: Hub and spoke with central login: this model is not suitable
for large scale management of access policies.

Credential translation. SAML identities are not compatible with OIDC
identities. 2nd, service (provider) federations and identity
(provider) federations have incompatible attribute dictionaries and
authentication protocols. In a sample 107 / 131 EUDAT implementation,
the credential translator consists of an OAuth2 authorization server,
and a certificate authority and federation database both part of 108 /
132. The user first authenticates with an identity provider, and then
may request a token from the translation OAuth2 server, which responds
by allowing the user to retrieve a token. The user then asks the
translation CA for a credential. The CA then retrieves user attributes
from the federation database, and provides the user with a short lived
x.509 credential. All of this has taken place in a security
environment. The user may now use her certificate to access services,
in a service area.

Section 3: Terms, Definitions, Acronyms (acronyms are on doc 2)
Authentication: a credential. Common authentication refers to the use
of a single credential to access multiple infrastructures, resources,
services, etc. (1)

Authorization: dealing with “what level of service is allowed to be
accessed, rather than who is accessing. (1) Service discovery: (53)

Web service: “software processes that enable fetching, adding,
editing, or deleting data.” (2)

Additional concepts to be added to this section – 
LoA calculation
(memon 99). Attribute provenance
(memon 71). Metadata schema
(memon 27 / 51) guest identity



\TODO{Lightening strike outage, server outage.}
