\FILE{section-registry.tex}

\subsection{Analytics Service Registry}
\label{sec:registry}

\paragraph*{Motivation.} 

The goal of a federated analytics service registry is to establish
federated registries to locate and consume analytics services with
persistent identifiers across organizations.

A service registry can serve as a public, private, or federated
registry. The first two properties define if the registry is public or
private. In case of a private registry proper security measures need
to be taken into account to govern access. Our framework does not make
any recommendations about the security framework chosen and it is up
to the implementer to specify it. In case of a federated registry,
more than one registry can be joined, to provide the user the
impression of a single registry.

Within the analytics services we distinguish two classes. The first
class are instantiated (running) services that are offered by a
service provider and allow direct reuse. The second class are library
providers that distribute analytics activity not as an instantiated
service, but as a source code library which can be deployed as a
service.


\TODO{use case} A user wants to find an analytics service and needs to
identify candidate services based on their descriptions and
features. A user wants to find services quickly and therefor expects
modern keyword search and taxonomy, faceted search, query
functionalities; as well as descriptions that facilitate location and
identification of relevant and appropriate analytics services, from the
registry.

The registry contains enough details to not only locate the service,
but also how to use it.

\paragraph*{Access Requirements.} 

\TODO{possibly change section to privacy requirements so that LoA
  and Authentication can be moved to separate section ?}

Public Analytics Service Registry. Public analytics discovery
services are intended to allow users to find publicly hosted
services. The information provided includes the provider, [x], and
[y], and / thus reduce users' efforts in locating relevant services.

\begin{description}

\item[Levels of Assurance (LoA) in User Identity.] Most readers should
  be familiar with functionality to {\em sign in with ORCHID, or
    Facebook} or something known to the user. In general identity
  management scenarios, this provision enables what is referred to as
  {\em guest identities}, which is useful for many users who are
  interested in invoking low level activities or less sensitive
  operations. With respect to federated service authentication and
  authorization, OIDC guest identities meet a low level of
  assurance. In contrast, users with higher LoAs are afforded
  permissions to perform to privileged activities or gain access to
  more sensitive xyz.

\item[Multi factor Authentication in User Identity.] A means for
  authenticating users via two or more types of
  authentication. An MFA instrument can elevate a user’s level of
  assurance profile. RAF and IGTF are examples of such assurance
  framework standards.  OpenID Connect, SAML, and X.509 are examples
  of services that expose interfaces for multiple authentication.

\item[Private Analytics Service Registry.] Analytics Services stored
  in private registries are only available to authenticated and
  authorized / member users. Private registries allow providers to
  build virtual organizations [/ VOMS] that advertise specialized
  services to its user community. In contrast to a public analytics
  registry, access controls in private registries are more
  restricted. In addition, different group privileges may restrict the
  visible analytics service to the user. See related sub sections on
  user identity and levels of user priveledge ...

\item[Federated Analytics Service Registry.] A user wants to make
  selection decisions regarding which service to use. Analytics
  service brokers / providers therefore offer a federated analytics
  service in which multiple services from multiple providers are
  included. Rather than having to visit multiple, separate providers'
  registries, the user can visit the federated registry of the
  analytics broker to lookup all potentially suitable services, via a
  single interface / browser. It may be expected that federated
  registries abstract the technical effort that casual users would
  experience during location and inspection of published analytics
  services.  Underlying analytics service registry technologies
  leverage cross - organization persistent identifiers, enhanced with
  information that the original service provider may not have
  available, and xyz. such "enrichment" may could include for example,
  cost comparisons, or (some type of) ratings from its user community.

\item[Enhanced Analytics Service Registry.] Both public and private
  registries my need to be enhanced by providing detailed information
  so the user has a better understanding of the offering and allows
  comparison to similar artefacts maintained and published in the
  registry. Information details may include for example, benchmark
  information, service level agreements, or cost measures such as
  carbon cost, or technical limitations such as storage access and
  availability for big data.
  
\end{description}

\paragraph*{Registry Namespace.}
To allow uniform integration of entries into a unified namespace, URLs
are used to distinguish the services. This includes two different
entities. Firstly, an entity that defines the code base of a
service. Such a code base could be for example hosted on publicly
accessible code repositories. Secondly, the namespace could include
instantiated analytics service endpoints that define a running
instance of an analytics service.

The attributes are listed in Table \ref{tab:reg}. Some attributes may
be optional and may be dependent on if they are deployed services, or
contain a library that may be deployed.


\begin{table}[htb]
\caption{Registry attributes \TODO{Somehow the width is wrong}}\label{tab:reg}
\resizebox{1.0\columnwidth}{!}{%
\begin{tabular}{p{3cm}p{6cm}p{0.5cm}p{0.5cm}}
Name	& Description	& \rot{\shortstack{Service\\ provider}}	& \rot{\shortstack{Library\\ provider}} \\
\hline
ID & 	UUID, globally unique &	\OK &	\OK \\
Name & 	Name of the service	& \OK	& \OK \\ 
Title & 	Human readable title &	\OK	& \OK \\
Public	& True if Public
(needs use case to delineate what pub private means) & 	\OK & \OK \\
Description	& Human readable short Description of the Service	& \OK & 	\OK \\
Endpoint &	The endpoint of the service	& \OK	&  \NA \\
List of Input Parameters &
	A list of parameters to the service. The parameters have each the form of name, function, type, value, access. The type indicates the data type. The access indicates if the parameter is a data stream, database, single value/function, event.
The function responds to a different function in case multiple are provided by the service.	& \OK	& \OK \\ 
List of Output Parameters 
  style (event, stream, data)
  value
  timestamp & 
	List of responses cast by the service. The responses have the form of function, name, type, value, access, timestamp. The type indicates the data type. The access indicates if the parameter is a data stream, database, single value/function, event.
The function responds to a different function in case multiple are provided by the service. & 	\OK 	& \OK \\
Version	& The version number or tag of the service	& \OK	& \OK \\
License	& The license description	& \OK	& \OK \\
Protocol & 	REST	& \OK	& \OK \\
Modified & 	Modification Timestamp	& \OK& \OK \\
Owner	& Name of the distributing entity, organization or individual. It could be a vendor.	& \OK	& O \\
Author &	Contact details of the people or organization responsible for the service	& O	& \OK \\
Tags &	Human readable common tags that are used to identify the service that are associated with the service	& O & O \\
Categories &	A category that this service belongs to (NLB, Finance, ...)	& O & O \\
Created	& date and time on which the analytics service was instantiated or created	instantiated	& \OK & \OK \\
Heartbeat &	State and timestamp of the last check when the service was active	& O & 	\NA \\
Documentation &	Link to a URL with detailed description of the service
Source	Link to the source code if available	& O & O \\
Specification &	Pointer to where specification schema is located	& O &  O \\
AdditionalMetadata	& Pointer to where additional is located including the one here.	& O &	O \\
SLA	& Serves level agreement including cost	& O 	& O \\
CachingInterval	&If a service is accessed a lot, the caching interval can be used to put a limitation on the Response with an LRU cache	& O &	\NA \\
DataIntegration &	In case of big data the data cannot be provided as a parameter to the analysis function. Instead, we need to provide the data as endpoint. However, often tata may need to be uploaded or can be downloaded. In this case this field provides the upload and download endpoints and the protocol to access the data	& O &	O \\
\hline
\end{tabular}
}
\OK = Required; O = Optional
\end{table}

\paragraph*{Benefits of a federated analytics service registry}

A service registry can publicize and improve end user access to data
from different sources, by overcoming some of the challenges inherent
in describing and surfacing document content and format. Publication,
and discovery of information resources are enriched with metadata
enabling the findability and reusability of a service supporting the
FAIR principle. While describing the interfaces and allow for the
instantiation or the reuse of already instantiated services we address
the accessibility and interoperability. With respect to analytics as a
service, end users should be able to find various analytic services
and similar services without having to individually search multiple
‘locations’ or databases, each built to operate on its own, unique
storage and retrieval constructs. Through these descriptions automated
service integration can be provisioned while targeting not only the
functionality involved, but also allowing service level considerations
to be addressed. Furthermore, such analytics services could provide
significant security implications such as the protection of a database
while only exposing a subset of approved analytics functions that are
executed on the data sets. This includes partial and controlled
sharing of data mashup that can be made available to the community and
registered to make reuse easier without everyone having to replicate
the service.

\clearpage
