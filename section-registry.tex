\begin{table}[H]
\caption{Registry attributes}\label{tab:reg}
\resizebox{\columnwidth}{!}{%
\begin{tabular}{p{3cm}p{11cm}p{0.5cm}p{0.5cm}}
Name	& Description	& \rot{\shortstack{Service\\ provider}}	& \rot{\shortstack{Library\\ provider}} \\
\hline
ID & 	UUID, globally unique &	\OK &	\OK \\
Name & 	Name of the service	& \OK	& \OK \\ 
Title & 	Human readable title &	\OK	& \OK \\
Public	& True if Public
(needs use case to delineate what pub private means) & 	\OK & \OK \\
Description	& Human readable short Description of the Service	& \OK & 	\OK \\
Endpoint &	The endpoint of the service	& \OK	&  N/A \\
List of Input Parameters &
	A list of parameters to the service. The parameters have each the form of name, function, type, value, access. The type indicates the data type. The access indicates if the parameter is a data stream, database, single value/function, event.
The function responds to a different function in case multiple are provided by the service.	& \OK	& \OK \\ 
List of Output Parameters 
  style (event, stream, data)
  value
  timestamp & 
	List of responses cast by the service. The responses have the form of function, name, type, value, access, timestamp. The type indicates the data type. The access indicates if the parameter is a data stream, database, single value/function, event.
The function responds to a different function in case multiple are provided by the service. & 	\OK 	& \OK \\
Version	& The version number or tag of the service	& \OK	& \OK \\
License	& The license description	& \OK	& \OK \\
Protocol & 	REST	& \OK	& \OK \\
Modified & 	Modification Timestamp	& \OK& \OK \\
Owner	& Name of the distributing entity, organization or individual. It could be a vendor.	& \OK	& O \\
Author &	Contact details of the people or organization responsible for the service	& O	& \OK \\
Tags &	Human readable common tags that are used to identify the service that are associated with the service	& O & O \\
Categories &	A category that this service belongs to (NLB, Finance, ...)	& O & O \\
Created	& date and time on which the analytics service was instantiated or created	instantiated	& \OK & \OK \\
Heartbeat &	State and timestamp of the last check when the service was active	& O & 	N/A \\
Documentation &	Link to a URL with detailed description of the service
Source	Link to the source code if available	& O & O \\
Specification &	Pointer to where specification schema is located	& O &  O \\
AdditionalMetadata	& Pointer to where additional is located including the one here.	& O &	O \\
SLA	& Serves level agreement including cost	& O 	& O \\
CachingInterval	&If a service is accessed a lot, the caching interval can be used to put a limitation on the Response with an LRU cache	& O &	N/A \\
DataIntegration &	In case of big data the data cannot be provided as a parameter to the analysis function. Instead, we need to provide the data as endpoint. However, often tata may need to be uploaded or can be downloaded. In this case this field provides the upload and download endpoints and the protocol to access the data	& O &	O \\
\hline
\end{tabular}
}
\OK = Required; O = Optional
\end{table}

\subsubsection{Benefits of a federated analytics service registry}

A service registry can publicize and improve end user access to data
from different sources, by overcoming some of the challenges inherent
in describing and surfacing document content and format. Publication,
and discovery of information resources are enriched with metadata
enabling the findability and reusability of a service supporting the
FAIR principle. While describing the interfaces and allow for the
instantiation or the reuse of already instantiated services we address
the accessibility and interoperability. With respect to analytics as a
service, end users should be able to find various analytic services
and similar services without having to individually search multiple
‘locations’ or databases, each built to operate on its own, unique
storage and retrieval constructs. Through these descriptions automated
service integration can be provisioned while targeting not only the
functionality involved, but also allowing service level considerations
to be addressed. Furthermore, such analytics services could provide
significant security implications such as the protection of a database
while only exposing a subset of approved analytics functions that are
executed on the data sets. This includes partial and controlled
sharing of data mashup that can be made available to the community and
registered to make reuse easier without everyone having to replicate
the service.

