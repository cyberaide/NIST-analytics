\section*{Report Production}

The production of this document is conducted to address the following needs (1) one-page executive summary, (2) a detailed specification, (3) use cases that support this document that may be hosted in separate documents. Such documents could follow the template as provided at  \cite{nist-bigdatawg}.

The draft document is currently managed in the cloud and access can be granted by contacting \url{laszewski@gmail.com}.

\section{Introduction}

\subsection{Background}

Discuss Focus Areas and Deliverables on Big Data Analytics with legacy analytics, machine/deep learning utilize our NIST Big Data Reference Architecture (NBD-RA) and NBD-RA Interfaces

\begin{enumerate}
\item[a.]	Exploration: Determine and document the level of 
  interest from industry, government, and academia in extending the NBDIF to develop scalable analytics as services that are reusable, deployable, and operational, regardless of the underlying computing environment.

\item[b.] Key Focus Areas

  \begin{enumerate}
  \item[b.1.] Compile and organize use cases, analytic services  
     from traditional statistical, AI/ML/DL, and emerging analytics application domains; identify and document technical requirements.
  \item[b.2.] Package analytic algorithms with well-defined input 
     and output parameters as service payloads that can be reusable, deployable, and operational across multi-cores, CPUs, and GPU computing platforms.
  \item[b.3.] Encapsulate service payload with a well-defined 
     format, interface, and end-to-end access control for the open and secure computing environment. 
  \item[b.4.] Establish federated registries to locate and 
     consume analytics services with persistent identifiers across organizations. 
  \item[b.5.] Provide resource management for application   
     orchestration and workflow between processes.
  \end{enumerate}
  
\end{enumerate}

\subsection{Scope and Objectives}

With Big Data’s compound annual growth rate at 61 percent and its ever-increasing deluge of information in the mainstream, the collective sum of world data will grow from 33 zettabytes (ZB, 1021) in 2018 to 175 ZB by 2025. The presence of such a rich source of information requires a massive analysis that can effectively bring about much insight and knowledge discovery. While previous work focused on developing a Big Data Reference Architecture. This work specifically focused on the definition of Analytics Services.

NBD-PWG is exploring how to extend NBDIF for packaging scalable analytics as services to meet the challenges of today’s information analytics. These services are intended to be reusable, deployable, and operational for Big Data, High Performance Computing, AI machine learning (ML), and deep learning (DL) applications, regardless of the underlying computing environment.

This document explores key focus areas and document level of interest from industry, government, and academia in extending the NBDIF to develop scalable analytics as services that are reusable, deployable, and operational, regardless of the underlying computing environment. 
\TODO{russell}{note that the 'string' reusable, deployable, and operational was also used in the previous ppg.}

The document is organized as follows and motivated by the tasks identified in the previous section.

\TODO{Gregor} the following will be reworked based on the new outline.

\begin{itemize}
    \item Section 42: Definitions and Concepts: Develop a brief list of definitions that can be used to improve communication between different interdisciplinary groups while allowing them to use the same language.

\item Section 53: Use Cases for Analytic Services: We plan to compile and organize use cases focusing on analytic services including traditional statistical, AI/ML/DL, and emerging analytics application domains. It will help identifying the meta- and technical requirements.

\item Section 5.24: Reusable Analytics Services. The main section will include the definition and conceptual architecture of reusable analytics services. This includes the following concerns that are organized as subsections: 
  
  \begin{itemize}
  \item Section ??: Security Considerations in Reusable Analytics Services: Here we explore a number of important security considerations related to reusable analytics services.

  \item Section ??: Data in Reusable Analytics Services: Here we explore a number of important issues  related to data that is used by the analytics services.

  \item Section  ??4: Package Analytic Algorithms as Service Payloads: Here we explore how to package analytic algorithms with well-defined input and output parameters as service payloads that can be reusable, deployable, and operational across multi-cores, CPUs, and GPU computing platforms.

  \item Section 5.5.24.4: Analytics Service Interfaces and Encapsulation: Here we explore a minimal set of services and their interfaces to be used as part of a generalized analytics framework. It includes to encapsulate the service payload with well-defined format, interface, and end-to-end access control for open and secure computing environment.
  
  \item Section 5.64.5: Service Registry: To communicate the existence and the features of the services to others service registries can be used.

  \item Section ??: Resource Management: Here we investigate and define a minimal set of resource management services and interfaces for application orchestration and workflow between processes.

  \end{itemize}

\item Section 65: Outreach Activity: In our outreach activity we investigate the inclusion and collaboration with other interested parties.

\end{itemize}

