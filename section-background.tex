\FILE{section-background.tex}

\section{Motivation}\label{s:background}

In this section, we explain the motivation  while summarizing
briefly, the different concepts constituting our work.

\subsection{Hybrid Analytics Services}

A {\em hybrid analytics service} combines the strength of analytics
services that are offered by providers in public, private, or
on-premise. It leverages them in order to provide ideally optimized
orchestration across them. This is not just limited to reducing cost,
but also to address security and privacy concerns when the data
analytics or the data to perform the analytics can not be hosted in
public clouds. Many of the major cloud providers such as AWS, Azure,
Google, IBM, Oracle, and others have made hybrid clouds a cornerstone
of their business model, with each of them mainly pushing their own
solutions. Recently, however, we see that the focus is no longer
offering just infrastructure but instead to provide services hiding the
cloud infrastructure entirely from the users.  This leads to vendors
also start providing hybrid analytics solutions that may integrate
multiple services offered by various providers resulting in solutions
with heterogenous service offerings. The integration of such services
provides challanges as each vendor may conform to different
integration solutions adressing public, private and on-premise
analytics services. In general, customers will benefit from a more
integrated approach to ease deployment and management concerns.

\subsection{Multi-Analytics Services}

Over the last two years, we have seen an explosion of analytics
services, mainly through the integration of AI and deep learning
services. High-level analytics services are being developed that hide the
complex infrastructure needed to embed not only services from one
vendor but multiple vendors. Hence we speak of {\em multi-analytics
  services}. These services can then be used in cooperation or
competition. We cooperate if services enhance each other, we compete
if a service is chosen over another service due to better service
level agreements. Through this interplay of the services, it is
beneficial to formalize this interoperability. In the case of competition,
we also need to be able to formulate a competing service that then
calls out other services to implement the desired analytics task.

\subsection{Enhancing the Big Data Reference Architecture}
\label{s:arch}

NIST has developed a Big Data Reference Architecture as part of
NBDIF\cite{nist-v6} and identified a number of use cases that motivate
it \cite{nist-v3}. We can leverage the effort
~\cite{nist-v1,nist-v2,nist-v3,nist-v4,nist-v5,nist-v6,nist-v7,nist-v8,nist-v9}
while formulating novel interoperability specifications that arise in
our effort that has not been addressed. While we previously focused
mostly on virtual machine management, this effort will enhance the
activities to include high-level coordinated service deployments and
utilization while leveraging containers.

\subsection{Identification of State-of-the-Art Data Analytics Patterns}

We have to ask why this opportunity can be addressed now to enable
this interplay?  It is clearly motivated by state-of-the-art data
analytics capabilities that only recently have become available.
Although government agencies have provided some of the most capable
high-end computing systems over the last years, they have integrated
specialized GPUs as well as container technologies to bring forward
new data analytics capabilities in these on-premise services. Industry
has provided advanced analysis capabilities for some time but only
recently they have achieved the maturity allowing reuse and
cooperation opportunities among all of them.

We like to provide a scenario that motivates our work. A health care
technician in the hospital uses voice commands to control an MRI
machine to take tomographic images. The images will be automatically
sent to a private analytics service to identify if the images contain
signs of COVID-19. Not only is one service consulted, but multiple to
assure that the best available algorithms are chosen. Once identified,
the image sare sent to a public analytics service (if the patient consents) in
order to improve the available images to improve the deep learning data
analysis. Previous images that have tested negative may be reanalyzed
with the newly improved models. If new cases are found based on
the improved analysis, health care providers are notified, and further
actions regarding the treatment plan by the supervising physician is
cast. As we can see, this example has all the ingredients that we need
to create a new generation of services that integrates on-premise
infrastructure, public and private services. The orchestration of the
services as well as a convenient interface. Automation of the workflow
of this use case example is explicitly stated.

Analyzing this example, we identify a mixture of services that utilize
on-premise infrastructure (the MRI), private and public services. A
variety of service patterns are used in concert to establish an analysis
pipeline targeted explicitly for this application use case. 

The goal of this work is to analyze the interoperability of such
service scenarios and identify such patterns motivating a vendor
neutral architecture that promotes reusable implementations to 
support aspects of similar use cases addressed by them.
