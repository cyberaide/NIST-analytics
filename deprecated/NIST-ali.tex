\section{Security analysis}


Multi Hybrid Cloud and edge computing Analytics as a service and ML are promising solutions that rapidly change how healthcare data is handled and create novel opportunities to diagnose and treat diseases. The enhancements leveraging large-scale data aggregation to train Analytics As Services models enable operational efficiency and diagnostic assistance tools in real-time on each care provider unit (I.e. Emergency (ER), Intensive Care Unit (ICU), and other care units select their specific needs.

Data Analytics as a service is the future for the healthcare industry to develop complex algorithms that can learn and resolve the behavior in diagnosing patient care.

However, data analytics as a service should be explored and credited by three major regulatory frameworks for any healthcare service.

a- The United States Food and Drug Administration (FDA), 
b- The European Union, and 
c- the International Medical Device Regulators 
Forum, respectively, show how they ensure the safety, effectiveness, and performance of Analytic-based applications. However, these regulatory bodies could be
 
Gaining knowledge and actionable insights from complex, high-dimensional, and heterogeneous biomedical data remains a key challenge in transforming health care. Data analytic services have emerged in modern biomedical research, including Electronic Health Records, imaging, -omics, and sensor data, which are complex, heterogeneous, poorly annotated, and generally unstructured. Traditional data mining and statistical learning approaches typically need feature analytics engineering to obtain adequate and more robust features from those data and then build prediction or clustering models—both steps in a scenario of complex data and insufficient domain knowledge. The latest advances in AI technologies provide new effective paradigms to obtain end-to-end learning models from complex data.
 
Legacy systems, Electronic Health Records (EHR), Medical Imaging (MI), and Genomics are healthcare data sources that drive clinical and operational decisions on the individual patient and population levels. 
Existing healthcare databases lack core designs. 
(a) to efficiently manage to explode big data analytics to improve traditional medical diagnostics methods with incredible speed and accuracy to improve patient diagnoses outcome.
(b) to enable population-scale data harmonization for research and computational capabilities that process and reduce costs associated with data transfers, 
(c) to assure alignment with accepted standards of medical ethics, 
(d) to assure data integrity, 
(e) to assure seamless and secure cloud access for system interoperability and data sharing, and 
(f) to assure digital rights by allowing each patient to fully and transparently control their data. 
In the past decade, in the healthcare world, where data is growing at unprecedented rates, a new strategy and a new set of tools are required to capture the promises of Big Data Analytics in the healthcare space. 




This use case has the implicit requirements needing the following
aspects to be addressed by the framework we develop.

\begin{enumerate}

\item{\bf AS vendor neutral cloud and computer service integration.} ...

  \begin{enumerate}
  \item {\bf AS in cloud.} ...
  \item {\bf AS in LCCF.} ...
  \item {\bf AS in microservices.} ...
  \end{enumerate}

\item{\bf AS architecture.} ...

  \begin{enumerate}
  \item{\bf AS vendor neutral interfaces.} ...
  \item{\bf AS REST.} ...
  \item{\bf AS layers such as interface, service layer, and provider layer.} ...
  \end{enumerate}

\item{\bf AS workflow.} ...

  \begin{enumerate}
  \item{\bf AS catalog and registry.} ...
  \item{\bf AS cooperation.} ...
  \item{\bf competition.} ...
  \item{\bf AS orchestrator.} ...
  \end{enumerate}


\item{\bf AS calculation.}

  \begin{enumerate}
  \item{\bf AS with DL.} ...
  \item{\bf AS data analytics.} ...
  \end{enumerate}

\item{\bf AS security.} ...

\end{enumerate}