\FILE{section-introduction.tex}

\newcommand{\WG}{\TODO{Wo: define name of wrking group}{WORKINGGROUP}}

\section{Introduction}


\subsection{Background}

With Big Data’s compound annual growth rate at 61 percent and its
ever-increasing deluge of information in the mainstream, the
collective sum of world data will grow from 33 zettabytes (ZB, 1021)
in 2018 to 175 ZB by 2025 \TODO{}{likely outdated by now. find out
  current trends and numbers}. The presence of such a rich source of
information requires a massive analysis that can effectively bring
about much insight and knowledge discovery. While previous work
focused on developing a Big Data Reference Architecture. This work
specifically focused on the definition of {\bf\em Analytics Services}.

We leverage activities conducted previously as part of the NIST Big
Data Reference Architecture (NBD-RA) and NBD-RA Interfaces.  However,
the work here targets explicitly {\bf\em Big Data Analytics} while integrationg
legacy analytics with machine and deep learning analytics. We will
focus on a service oriented framework.


\subsection{Scope and Objectives}

NBD-PWG\footnote{\TODO{}{introduce in the background section. THe
background section has been integrated here. check valitity and fix
somehow.}} is exploring how to extend
NBDIF \footnote{\TODO{}{introduce in the background section. THe
background section has been integrated here. check valitity and fix
somehow.}} for packaging scalable analytics as services to meet the
challenges of today’s information analytics. These services are
intended to be reusable, deployable, and operational for Big Data,
High Performance Computing, AI machine learning (ML), and deep
learning (DL) applications, regardless of the underlying computing
environment.

This document explores key focus areas and document level of interest
from industry, government, and academia in extending the NBDIF to
develop scalable analytics as services that are reusable, deployable,
and operational, regardless of the underlying computing environment.
\TODO{russell}{note that the 'string' reusable, deployable, and
  operational was also used in the previous ppg.}

The document is organized as follows and motivated by the tasks
identified in the previous section. Each section is augmented with the
key area it contributes to.

\begin{itemize}
  
\item Section \ref{sec:definitions}: Definitions and Concepts: Develop
  a brief list of definitions that can be used to improve
  communication between different interdisciplinary groups while
  allowing them to use the same language.

\item Section \ref{sec:usecases}: Use Cases for Analytic Services:
  A compilation and organization of use cases focusing on analytic
  services including traditional statistical, AI/ML/DL, and emerging
  analytics application domains. It will help identifying the meta-
  and technical requirements.

\item Section \ref{sec:defining}: Defining and Finding Reusable Analytics
  Services. This section 
  will include the definition and conceptual architecture of reusable
  analytics services. This includes the following concerns that are
  organized as subsections.

 \begin{itemize}
 
    \item Section \ref{sec:fair}: Adaptation of the FAIR principle to
       supprt an Analytics Service FAIR Principle.

    \item Section \ref{sec:catalog}: Service Catalog: To communicate
      the existence of the services to others service registries can
      be used.

    \item Section \ref{sec:registry}: Service Registry: To communicate
      the the features of the services to others service registries
      can be used.

\end{itemize}

\item Section \ref{sec:federation} Service Analytics Federation: To leverage multiple
existing services federated services can be used to integarte them.

\item Section \ref{sec:data}: Data in Reusable Analytics Services:
  Here we explore a number of important issues related to data that is
  used by the analytics services.

\item Section \ref{sec:package}: Package Analytic Algorithms as
  Service Payloads: Here we explore how to package analytic algorithms
  with well-defined input and output parameters as service payloads
  that can be reusable, deployable, and operational across
  multi-cores, CPUs, and GPU computing platforms.

\item Section \ref{sec:interfaces}: Analytics Service Interfaces and
  Encapsulation: Here we explore a minimal set of services and their
  interfaces to be used as part of a generalized analytics
  framework. It includes to encapsulate the service payload with
  well-defined format, interface, and end-to-end access control for
  open and secure computing environment.
  
\item Section \ref{sec:resources}: Resource Management: Here we
  investigate and define a minimal set of resource management services
  and interfaces for application orchestration and workflow between
  processes.

\item Section \ref{sec:security}: Security Considerations in Reusable
  Analytics Services: Here we explore a number of important security
  considerations related to reusable analytics services.  

\item Section \ref{sec:outreach}: Outreach Activity: In our outreach
  activity we investigate the inclusion and collaboration with other
  interested parties.

\end{itemize}

