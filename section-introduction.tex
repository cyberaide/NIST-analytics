\FILE{section-introduction.tex}

\newcommand{\WG}{\TODO{Wo: define name of wrking group}{WORKINGGROUP}}

\section{Introduction}


\subsection{Background}

With Big Data's compound annual growth rate at 61 percent and its
ever-increasing deluge of information in the mainstream, the
collective sum of world data will grow from 33 zettabytes (ZB, 1021)
in 2018 to 175 ZB by 2025 \TODO{likely outdated by now. find out
  current trends and numbers}. The presence of such a rich source of
information requires a massive analysis that can effectively bring
about much insight and knowledge discovery. While previous work
focused on developing a Big Data Reference Architecture. This work
specifically focused on the definition of {\bf\em Analytics Services}.

We leverage activities conducted previously as part of the NIST Big
Data Reference Architecture (NBD-RA) and NBD-RA Interfaces.  However,
the work here targets explicitly {\bf\em Big Data Analytics} while integrationg
legacy analytics with machine and deep learning analytics. We will
focus on a service oriented framework.


\subsection{Scope and Objectives}

NBD-PWG\footnote{\TODO{introduce in the background section. THe
background section has been integrated here. check valitity and fix
somehow.}} is exploring how to extend
NBDIF \footnote{\TODO{introduce in the background section. THe
background section has been integrated here. check valitity and fix
somehow.}} for packaging scalable analytics as services to meet the
challenges of today's information analytics. These services are
intended to be reusable, deployable, and operational for Big Data,
High Performance Computing, AI machine learning (ML), and deep
learning (DL) applications, regardless of the underlying computing
environment.

This document explores key focus areas and document level of interest
from industry, government, and academia in extending the NBDIF to
develop scalable analytics as services that are reusable, deployable,
and operational, regardless of the underlying computing environment.
\TODO{Russel: note that the 'string' reusable, deployable, and
  operational was also used in the previous ppg.}


