\FILE{section-summary}

\section{Introduction}
\label{sec:summary}


Analytics as a service has become a multi-billion dollar opportunity
for the industry. Research institutions and private companies offer a number
of competing but also collaborating services. While in the past
emphasis has been placed on the hybrid multi-cloud infrastructure
services, the focus is now [significantly] shifting to offering {\em
analytics [as] services} and not just infrastructure, to customers and
researchers. As such, it is important to identify how researchers and
industry can interoperate with services offered by various service
providers. Analog to the terminology in cloud computing, we introduce
the terms {\em hybrid} analytics services to include services run by
remote service providers or by an organization on local resources. 
In
addition, we use the term {\em multi} analytics services to indicate
that multiple services from potentially multiple service providers
work in concert to offer a new capability released to users as
analytics services. While applying such services to data, we term the
combination of such services as {\em hybrid multi-services data
analytics} services. If properly put in place, the resulting service
accelerates new solutions offered by industry and research as new
services with a combined new functionality which can be {\em reused}.

To achieve this goal, requires platform-independent
interoperability.  To assure that the pathway to leverage these hybrid
and multi-analytics services is kept at a high level while at the same
time exposing enough details, we need to support intelligent
decision-making as part of the service orchestration. Services must be
chosen to fulfill a set of {\em analytics service} level requirements
posed by the users. It is of particular interest how we can formulate
hybrid analytics services and multi-analytics services offered by
different providers that provide other features. The user needs to
specify this via a simple analytics service provider independent
specification.

Our data analysis intends to be capable of determining which service is
suitable or chosen based on its requirements and to what degree
reusability is offered while replicating the analysis across different
services. Hence we will work towards a {\em ``Reusable Hybrid
Multi-Services Data Analytics Framework''}. This results in a research
platform that allows the creation of of an integrated application
platform benefiting from reusable hybrid analytics services.


By integrating such services, we will be able to significantly impact
data analytics while leveraging not only one vendor's implementation
but by promoting the reuse services via a {\em many-vendors}
approach. Not only that, but we will also allow the interplay between
different approaches while offering a uniform specification platform.
Because we target the topic of this interplay, the effort has been
done in collaboration with the NIST Information Access Division (IAD)
in the NIST Big Data Working Group.

The paper is structured as follows. First we present the motivation
leading up to this work (Section~\ref{s:background}), followed by a
discussion about requirements that we derived by analysing a number of
complex usecases. Next we present our architectural approach, which is
based on lessons learned from the requirements we have gathered and
lessons learned during our implementation (Section~\ref{s:arch}. We
present an architectural design capable of supporting the needs we
have identified.  Finally, we present our conclusions
(Section~\ref{sec:conlcusion}).
