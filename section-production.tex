\FILE{section-production.tex}

\newcommand{\WG}{OUR WORKING GROUP}

\section*{Report Production}


This document is conducted as part of the NIST \WG. The working group
holds public meetings regularly Tuesdays at 1 pm. The
working group is public, and anyone can join that is interested in
contributing to this document and bringing it to completion. We will
be using GitHub to coordinate this work. Although the work is done in
a public working group, we had one member of the group asking to keep
the document private. As such, you need to join our meeting group first
before we grant you access to this document. You can join the working
group by contacting Wo Chang at \verb|wchang@nist.gov|.


The production of this document is conducted to address the following
needs

\begin{enumerate}
  \item a one-page executive summary, 

  \item a detailed specification,

  \item use cases that support this document that may be hosted in
    separate documents. Such documents could follow the template as
    provided at \cite{nist-bigdatawg}.

\end{enumerate}


\parindent0pt We are using the following tools to manage the completion of
the document:

\begin{itemize}

\item
  \href{https://www.overleaf.com/project/619ba513e4aade4400e06df8}{an
    online document $\rightarrow$} we work on in Overleaf. Oevrleaf
  sinks to GitHub are done however only on a best effort basis. SO
  before you start working, make sure you pull in overleaf from
  GitHub.

\item \href{https://github.com/orgs/cyberaide/projects/1}{a task list
  $\rightarrow$} in GitHub,

\item \href{https://github.com/orgs/cyberaide/}{a versioned document
  source $\rightarrow$} in GitHub,

\item \href{https://nist-analytics.slack.com}{a slack channel
  $\rightarrow$} to increas the communication outside of the Working
  group meetings and to coordinate task.

\item
  \href{https://github.com/cyberaide/NIST-analytics/raw/main/NIST-analytics.pdf}{a
    regularly updated PDF document} with editing links pointing to the
  GItHub versioned documents. Please note that for the most recent
  document you will have to use GitHub or Overleaf. 
  
\end{itemize}

Once you have joined the Working group, you can get access to the
document directories by contacting \url{laszewski@gmail.com}.

\begin{quote}

{\em Please note that we stopped using Word for the document
  management contributors consistently used different templates and
  formatting rules resulting in unsustainable multi hourlong cleanups
  every week instead of being able to focus on content. For this
  reason, we will not switch back to Word. However, a section could be
  written in markdown, which can easily be integrated into the
  document. Formatting should be kept at a minimum.}

\end{quote}
