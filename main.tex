%%%%%%%%%%%%%%%%%%%%%%%%%%%%%%%%%%%%%%%%%%%%%%%%%%%%%%%%%%%%%%%%%%%%%%%%
%     LaTeX source code to approximate a NIST Technical report
%	  Instructions for authors: tinyurl.com/techpubsnist 
%	DOI watermark will be added on final PDF
% 	Developed by Gregor von Laszewski based on the 1200 template provided by K. Miller, kmm5@nist.gov 
%	Last updated: 22-November-2021
%%%%%%%%%%%%%%%%%%%%%%%%%%%%%%%%%%%%%%%%%%%%%%%%%%%%%%%%%%%%%%%%%%%
\documentclass[12pt]{article}
\usepackage{amsmath}
\usepackage{amsfonts}   % if you want the fonts
\usepackage{amssymb}    % if you want extra symbols
\usepackage{graphicx}   % need for figures
\usepackage{xcolor}
\usepackage{bm}
\usepackage{secdot}		
\usepackage{mathptmx}
\usepackage{float}
\usepackage[utf8]{inputenc}
\usepackage{textcomp}
\usepackage[hang,flushmargin,bottom]{footmisc} % footnote format
\usepackage{comment}
\usepackage[shortlabels]{enumitem}

\usepackage{titlesec}
\titleformat{\section}{\normalsize\bfseries}{\thesection.}{1em}{}	% required for heading numbering style
\titleformat*{\subsection}{\normalsize\bfseries}

\usepackage{tocloft}	% change typeset, titles, and format list of appendices/figures/tables
\renewcommand{\cftdot}{}	
\renewcommand{\contentsname}{Table of Contents}
\renewcommand{\cftpartleader}{\cftdotfill{\cftdotsep}} % for parts
\renewcommand{\cftsecleader}{\cftdotfill{\cftdotsep}}
\renewcommand\cftbeforesecskip{\setlength{4pt}{}}
\addtolength{\cftfignumwidth}{1em}
\renewcommand{\cftfigpresnum}{\figurename\ }
\addtolength{\cfttabnumwidth}{1em}
\renewcommand{\cfttabpresnum}{\tablename\ }
\setlength{\cfttabindent}{0in}    %% adjust as you like
\setlength{\cftfigindent}{0in} 

\usepackage{enumitem}         % to control spacing between bullets/numbered lists

\usepackage[numbers,sort&compress]{natbib} % format bibliography 
\renewcommand{\bibsection}{}
\setlength{\bibsep}{0.0pt}

\usepackage[hidelinks]{hyperref}
\hypersetup{
	colorlinks = true,
urlcolor ={blue},
citecolor = {.},
linkcolor = {.},
anchorcolor = {.},
filecolor = {.},
menucolor = {.},
runcolor = {.}
pdftitle={},%%put title here to auto-fill properties of the PDF
pdfsubject={},%%put abstract here
pdfauthor={}, %%put author list here
pdfkeywords={} %%put keywords here
}
\urlstyle{same}

\usepackage{epstopdf} % converting EPS figure files to PDF

\usepackage{fancyhdr, lastpage}	% formatting document, calculating number of pages, formatting headers
\setlength{\topmargin}{-0.5in}
\setlength{\headheight}{39pt}
\setlength{\oddsidemargin}{0.25in}
\setlength{\evensidemargin}{0.25in}
\setlength{\textwidth}{6.0in}
\setlength{\textheight}{8.5in}

\usepackage{caption} % required for Figure labels
\captionsetup{font=small,labelfont=bf,figurename=Fig.,labelsep=period,justification=raggedright} 

\usepackage[most]{tcolorbox}
\usepackage{xcolor}

\usepackage{pifont}
\newcommand*\rot{\rotatebox{90}}
\newcommand*\OK{\ding{51}}
\usepackage[table]{xcolor}
\usepackage{forest}
\usetikzlibrary{arrows.meta,shapes,positioning,shadows,trees}
%
\tikzset{
    basic/.style  = {draw, text width=2cm, drop shadow, font=\sffamily, rectangle},
    root/.style   = {basic, rounded corners=2pt, thin, align=center,
                     fill=green!10},
    onode/.style = {basic, thin, rounded corners=2pt, align=center, fill=green!30,text width=3cm,},
    tnode/.style = {basic, thin, align=left, fill=pink!20, text width=6.5em},
    edge from parent/.style={draw=black, edge from parent fork right}
}
%
\usepackage{titlesec}
\setcounter{tocdepth}{10}  

\setcounter{secnumdepth}{10}

\usepackage{markdown}

\newcommand{\pubnumber}{1500-XX}
\newcommand{\DOI}{https://doi.org/10.6028/NIST.SP.1500-XX}
\newcommand{\monthyear}{November 2021}
\newcommand{\TITLE}{NIST Reusable Analytics Service Framework\\
Volume 1, Overview}
\newcommand{\AUTHORS}{
\large Gregor von Laszewski\\
\large Wo Chang\\
\large Russell C. Reinsch\\
\large Geoffrey C. Fox\\
\vfill}
\usepackage{lineno}
%%%%%%%%%%%%%%%%%%%%%%%%%%%%%%%%%%%%%%%%%%%%%%%%%%%%%%%%%%%%%%%%%%%%
%   	BEGIN DOCUMENT 
%%%%%%%%%%%%%%%%%%%%%%%%%%%%%%%%%%%%%%%%%%%%%%%%%%%%%%%%%%%%%%%%%%%%
\begin{document}

	\urlstyle{rm} % Format style of \url   
	\begin{titlepage}
		\begin{flushright}
        \LARGE{\textbf{NIST Special Publication \pubnumber}}\\
        \vfill
        \Huge{\textbf{\TITLE}}\\
        \vfill
        \AUTHORS
        \normalsize This publication is available free of charge from:\\
        \DOI\\
        \vfill

        \includegraphics[width=0.3\linewidth]{format/NIST-logo.eps}\\ 

        \end{flushright}
    \end{titlepage}

\begin{titlepage}

\begin{flushright}
\LARGE{\textbf{NIST Special Publication \pubnumber}}\\
\vfill 
\Huge{\textbf{\TITLE}}\\
\vfill
%%%%%%%%%%%%%%%%%%%%%%%%%%%%%%%%%%%%%%%%%%%%%%%%%%%%%%%%%%%%%%%%%%%%
%	Author Order and Grouping. Always identify the primary author/creator first (s/he does not have to be a NIST author). For publications with multiple authors, group authors by their organizational affiliation. The organizational groupings and the names within each grouping should generally be ordered by decreasing level of contribution.
%	For non-NIST authors, list their city and state below their organization name.
%	For NIST authors, include the Division and Laboratory names (but do not include their city and state).
%%%%%%%%%%%%%%%%%%%%%%%%%%%%%%%%%%%%%%%%%%%%%%%%%%%%%%%%%%%%%%%%%%%%
\normalsize Gregor von Laszewski\\
\textit{University of Virginia, Inidiana University, laszewski@gmail.com}\\
\vspace{12pt}

Wo Chang\\
\textit{NIST ITL, wchang@nist.gov}\\
\vspace{12pt}

\normalsize Russell C. Reinsch \\
\textit{MSI, russell@mcleansystemsintegration.com}\\
\vspace{12pt}

\normalsize Geoffrey C. Fox\\
\textit{Indiana University, gcfexchange@gmail.com}
\vspace{12pt}

\vfill
%%%%%%%%%%%%%%%%%%%%%%%%%%%%%%%%%%%%%%%%%%%%%%%%%%%%%%%%%%%%%%%%%%%%
%   DOI Statement - automated
%%%%%%%%%%%%%%%%%%%%%%%%%%%%%%%%%%%%%%%%%%%%%%%%%%%%%%%%%%%%%%%%%%%%
\normalsize This publication is available free of charge from:\\
\DOI\\
\vfill
%%%%%%%%%%%%%%%%%%%%%%%%%%%%%%%%%%%%%%%%%%%%%%%%%%%%%%%%%%%%%%%%%%%%
%   Date - Month and Year - automated
%%%%%%%%%%%%%%%%%%%%%%%%%%%%%%%%%%%%%%%%%%%%%%%%%%%%%%%%%%%%%%%%%%%%
\normalsize \monthyear
\vfill
%%%%%%%%%%%%%%%%%%%%%%%%%%%%%%%%%%%%%%%%%%%%%%%%%%%%%%%%%%%%%%%%%%%%
%  Department of Commerce LOGO - leave as-is
%%%%%%%%%%%%%%%%%%%%%%%%%%%%%%%%%%%%%%%%%%%%%%%%%%%%%%%%%%%%%%%%%%%%	

\includegraphics[width=0.18\linewidth]{format/DoC-logo.eps}\\ 
\vfill
%%%%%%%%%%%%%%%%%%%%%%%%%%%%%%%%%%%%%%%%%%%%%%%%%%%%%%%%%%%%%%%%%%%%
%  Department of Commerce & NIST Leadership 
%	will be updated as changes occur
%%%%%%%%%%%%%%%%%%%%%%%%%%%%%%%%%%%%%%%%%%%%%%%%%%%%%%%%%%%%%%%%%%%%
\footnotesize U.S. Department of Commerce\\ 
\textit{Gina M. Raimondo, Secretary}\\
\vspace{10pt}
National Institute of Standards and Technology\\ 
\hspace*{-3cm}\textit{James K. Olthoff, Performing the Non-Exclusive Functions and Duties of the Under Secretary of Commerce \\
for Standards and Technology \& Director, National Institute of Standards and Technology} 
\end{flushright}
\end{titlepage}

\linenumbers

\begin{titlepage}
%%%%%%%%%%%%%%%%%%%%%%%%%%%%%%%%%%%%%%%%%%%%%%%%%%%%%%%%%%%%%%%%%%%%
%   Disclaimer/CODEN page - required
%%%%%%%%%%%%%%%%%%%%%%%%%%%%%%%%%%%%%%%%%%%%%%%%%%%%%%%%%%%%%%%%%%%%
% \begin{flushright}
\footnotesize  

%%%%%%%%%%%%%%%%%%%%%%%%%%%%%%%%%%%%%%%%%%%%%%%%%%%%%%%%%%%%%%%%%%%%
%   This section automated - do not change
%%%%%%%%%%%%%%%%%%%%%%%%%%%%%%%%%%%%%%%%%%%%%%%%%%%%%%%%%%%%%%%%%%%%
\begin{flushright}
\normalsize \textbf{National Institute of Standards and Technology Special Publication \pubnumber\\ 
Natl. Inst. Stand. Technol. Spec. Publ. \pubnumber, \pageref{LastPage} pages (\monthyear)} \\
\textbf{CODEN: TBD}\\
\vspace{12pt}
\textbf{This publication is available free of charge from: \DOI}

\end{flushright}


Certain commercial entities, equipment, or materials may be identified in this document to describe an experimental procedure or concept adequately. Such identification is not intended to imply recommendation or endorsement by NIST, nor is it intended to imply that the entities, materials, or equipment are necessarily the best available for the purpose. 

There may be references in this publication to other publications currently under development by NIST in accordance with its assigned statutory responsibilities. The information in this publication, including concepts and methodologies, may be used by federal agencies even before the completion of such companion publications. Thus, until each publication is completed, current requirements, guidelines, and procedures, where they exist, remain operative. For planning and transition purposes, federal agencies may wish to closely follow the development of these new publications by NIST. 

Organizations are encouraged to review all publications during public comment periods and provide feedback to NIST. All NIST publications are available at http://www.nist.gov/publication-portal.cfm.



\vspace{12pt}
\vspace{12pt}
NIST Special Publication series 1500 is intended to capture external perspectives related to NIST standards, measurement, and testing-related efforts. These external perspectives can come from industry, academia, government, and others. These reports are intended to document external perspectives and do not represent official NIST positions.

\bigskip
\bigskip
\bigskip

\begin{quote}
Comments on this publication may be submitted to 

National Institute of Standards and Technology \\
Attn: Wo Chang, Information Technology Laboratory\\
100 Bureau Drive (Mail Stop 8900)\\ Gaithersburg, MD 20899-8930\\
Email: SP1500comments@nist.gov
\end{quote}

\vfill

\end{titlepage}
%%%%%%%%%%%%%%%%%%%%%%%%%%%%%%%%%%%%%%%%%%%%%%%%%%%%%%%%%%%%%%%%%%%%
%   Start front matter - page number starts with "i"
%%%%%%%%%%%%%%%%%%%%%%%%%%%%%%%%%%%%%%%%%%%%%%%%%%%%%%%%%%%%%%%%%%%%
\section*{Foreword}
\pagenumbering{roman}
\normalsize Delete if not applicable\\
\section*{Preface}
\normalsize Delete if not applicable\\

\begin{abstract}

\FILE{section-abstract.tex}

Over the last several years, the computation landscape for conducting
data analytics has completely changed. While in the past a lot of the
activities have been undertaken in isolation by companies and research
institutions, today's infrastructure constitutes a wealth of services
offered by a variety of providers that offer opportunities for reuse,
interactions whil leveraging service colaboration, and service
cooperation.

In this document, we focus on expanding analytics services to
developing a framework for reusable hybrid multi-service data
analytics. It includes (a) a short technology review that explicitly
targets the intersection of hybrid multi-provider analytics services
(b) a small motivation based on usecases with looked at (c) enhancing
the concepts of services to showcase how hybrid, as well as
multi-provider services, can be integrated and reused via the proposed
framework, (d) address analytics service composition, and (e)
integrate container technologies to achieve state-of-the-art analytics
service deployment capabilities.


\end{abstract}


\section*{Key words}
\normalsize 
Adoption; barriers; implementation; interfaces; market maturity; organizational maturity; project maturity; system modernization.\\

\pagebreak
%%%%%%%%%%%%%%%%%%%%%%%%%%%%%%%%%%%%%%%%%%%%%%%%%%%%%%%%%%%%%%%%%%%%
%   Table of Contents is required
% 	List of Tables & Figures required if more than 5 tables/figures
%%%%%%%%%%%%%%%%%%%%%%%%%%%%%%%%%%%%%%%%%%%%%%%%%%%%%%%%%%%%%%%%%%%%
\tableofcontents 
	
\listoftables

\listoffigures

\pagebreak

\section*{Glossary}
Delete if not applicable\\
\pagebreak


%%%%%%%%%%%%%%%%%%%%%%%%%%%%%%%%%%%%%%%%%%%%%%%%%%%%%%%%%%%%%%%%%%%%
%   Start body of text - page number starts with "1"
%%%%%%%%%%%%%%%%%%%%%%%%%%%%%%%%%%%%%%%%%%
%%%%%%%%%%%%%%%%%%%%%%%%%%

\pagenumbering{\arabic}

\section{Executive Summary}
\newpage

\section*{Report Production}

The production of this document is conducted to address the following needs (1) one-page executive summary, (2) a detailed specification, (3) use cases that support this document that may be hosted in separate documents. Such documents could follow the template as provided at  \cite{nist-bigdatawg}.

The draft document is currently managed in the cloud and access can be granted by contacting \url{laszewski@gmail.com}.

\section{Introduction}

\subsection{Background}

Discuss Focus Areas and Deliverables on Big Data Analytics with legacy analytics, machine/deep learning utilize our NIST Big Data Reference Architecture (NBD-RA) and NBD-RA Interfaces

\begin{enumerate}
\item[a.]	Exploration: Determine and document the level of 
  interest from industry, government, and academia in extending the NBDIF to develop scalable analytics as services that are reusable, deployable, and operational, regardless of the underlying computing environment.

\item[b.] Key Focus Areas

  \begin{enumerate}
  \item[b.1.] Compile and organize use cases, analytic services  
     from traditional statistical, AI/ML/DL, and emerging analytics application domains; identify and document technical requirements.
  \item[b.2.] Package analytic algorithms with well-defined input 
     and output parameters as service payloads that can be reusable, deployable, and operational across multi-cores, CPUs, and GPU computing platforms.
  \item[b.3.] Encapsulate service payload with a well-defined 
     format, interface, and end-to-end access control for the open and secure computing environment. 
  \item[b.4.] Establish federated registries to locate and 
     consume analytics services with persistent identifiers across organizations. 
  \item[b.5.] Provide resource management for application   
     orchestration and workflow between processes.
  \end{enumerate}
  
\end{enumerate}

\subsection{Scope and Objectives}

With Big Data’s compound annual growth rate at 61 percent and its ever-increasing deluge of information in the mainstream, the collective sum of world data will grow from 33 zettabytes (ZB, 1021) in 2018 to 175 ZB by 2025. The presence of such a rich source of information requires a massive analysis that can effectively bring about much insight and knowledge discovery. While previous work focused on developing a Big Data Reference Architecture. This work specifically focused on the definition of Analytics Services.

NBD-PWG is exploring how to extend NBDIF for packaging scalable analytics as services to meet the challenges of today’s information analytics. These services are intended to be reusable, deployable, and operational for Big Data, High Performance Computing, AI machine learning (ML), and deep learning (DL) applications, regardless of the underlying computing environment.

This document explores key focus areas and document level of interest from industry, government, and academia in extending the NBDIF to develop scalable analytics as services that are reusable, deployable, and operational, regardless of the underlying computing environment. 
\TODO{russell}{note that the 'string' reusable, deployable, and operational was also used in the previous ppg.}

The document is organized as follows and motivated by the tasks identified in the previous section.

\TODO{Gregor} the following will be reworked based on the new outline.

\begin{itemize}
    \item Section \ref{??}: Definitions and Concepts: Develop a brief list of definitions that can be used to improve communication between different interdisciplinary groups while allowing them to use the same language.

\item Section \ref{??}: Use Cases for Analytic Services: We plan to compile and organize use cases focusing on analytic services including traditional statistical, AI/ML/DL, and emerging analytics application domains. It will help identifying the meta- and technical requirements.

\item Section \ref{??}: Reusable Analytics Services. The main section will include the definition and conceptual architecture of reusable analytics services. This includes the following concerns that are organized as subsections: 
  
  \begin{itemize}
  \item Section \ref{??}: Security Considerations in Reusable Analytics Services: Here we explore a number of important security considerations related to reusable analytics services.

  \item Section \ref{??}: Data in Reusable Analytics Services: Here we explore a number of important issues  related to data that is used by the analytics services.

  \item Section  \ref{??}: Package Analytic Algorithms as Service Payloads: Here we explore how to package analytic algorithms with well-defined input and output parameters as service payloads that can be reusable, deployable, and operational across multi-cores, CPUs, and GPU computing platforms.

  \item Section \ref{??}: Analytics Service Interfaces and Encapsulation: Here we explore a minimal set of services and their interfaces to be used as part of a generalized analytics framework. It includes to encapsulate the service payload with well-defined format, interface, and end-to-end access control for open and secure computing environment.
  
  \item Section \ref{??}: Service Registry: To communicate the existence and the features of the services to others service registries can be used.

  \item Section \ref{??}: Resource Management: Here we investigate and define a minimal set of resource management services and interfaces for application orchestration and workflow between processes.

  \end{itemize}

\item Section \ref{s:outreach}: Outreach Activity: In our outreach activity we investigate the inclusion and collaboration with other interested parties.

\end{itemize}



\section{Terms and Definitions (we could place this as Appendix as needed)}



\section{Use Cases for Analytics Services}


\subsection{Use Case Development Process}

To specify use cases for our analytics framework, we encourage contributors to contact us and provide us with their high-level descriptions of their use cases. The use cases should be focusing on highlighting one or multiple aspects of the features related to analytics frameworks. While inspecting the various features we intend to collect and analyze them in various contexts that are relevant for analytics users. The lessons learned from this analysis are to be integrated into this document in order to formulate a comprehensive vendor neutral analytics framework.
Use cases can be formulated in various format but should include diagrams that make them easy to comprehend as well as allowing the reader to extract the specific analytical aspects. Such diagrams can include functional cross diagrams, process diagrams and others.
Use cases should especially address the use of metadata describing the functional and the data related properties. This includes metadata related to time, space, exchange/protocols, privacy, and security related aspects.
A functional description of the use case is to be included as a subsection called Functionalities and Activities. This section is mirroring our experience with documenting use cases as part of the Big Data Application Provider of NBDIF Reference Architecture. Hence, we assume the following draft form for a use case:

\begin{description}
\item[Title:] 		Title of the use case
\item[Contributor:] 	The list co contributors
\item[Description:] 	One to two sentences about major functionalities and activities with respect to the sample cross-functional diagram
	
	\begin{description}
	\item[Cross-Functional Diagram:]
		Inclusion of a cross functional Diagram, alternatively other diagrams could be 			chosen.
	\item[Functionality Activities:]
	\begin{enumerate}
        \item Activity \#1 – description…
        \item ...
        \item Activity \#n – description…
    	\item Use Case Summaries
    \end{enumerate}
\end{description}
\end{description}

\begin{figure}[htb]
    \centering
    \includegraphics[width=1.0\textwidth]{images/cross-functional-diagram.png}
    \caption{Cross-functional Diagram}
    \label{fig:cross-functional-diagram}
\end{figure}


Here we will list use case summaries and if available point to specific publications on the NBDIF Web page that include more details. The expectation of this section is to
 
\begin{itemize}
\item	Provide an overview of use cases that motivate this document
\item	It will summarize requirements that we obtain from these use cases that influence how we proceed.
\end{itemize}

As a result, we identify how they fit into the workflow of data analytics. This includes the description of a subset of functionality that is used in general by data analytics.  In particular, it described the relationship between input and output of data analytics components and interfaces. 

The use case summaries are expected to be available through the BigDataWG Web page at 


This includes the following examples:

\begin{enumerate}
\item	\href{https://bigdatawg.nist.gov/_uploadfiles/M0701_v1_2020102001.docx}{M0701} – Use case template
\item	\href{https://bigdatawg.nist.gov/_uploadfiles/M0702_v1_2020102002.pdf}{M0702} - Numeric weather prediction
\item	\href{https://bigdatawg.nist.gov/_uploadfiles/M0703_v1_2020102003.pdf}{M0703} - HVAC Heat ventilation and air conditioning 
\end{enumerate}



\subsection{Use Case 1: Title}
\subsection{Use Case 2: Title}


\section{Technical Requirements}

\section{Specification Paradigm}

\subsection{Relevant Standards and Best Practices}
\subsection{Standard Technology Gap Analysis}

\section{Analytics Paradigms}

\subsection{Introduction}

\subsection{Paradigm Landscape}

\subsection{Paradigms}
\subsubsection{Realtime Analytics} 
\subsubsection{Datalake Analytics} 
\subsubsection{AI/ML/DL  Analytics} 
\subsubsection{Legacy Analytics}
\subsubsection{Cloud Service Analytics}

\subsubsection{Edge Service Analytics}
\subsubsection{Container Service Analytics}
\subsubsection{Cluster Service Analytics}
\subsubsubsection{Batch Analytics} 
\subsubsection{Parameter Study Analytics}
\subsubsection{Workflow and Combinations}






\section{Specification}


\subsection{AS-FAIR-DO: Analytics Service FAIR Principle}

To project easy reusability, we strive towards the implementation of the AS-FAIR-DO principle for analytics services. The FAIR principle is typically applied to data and as such we can apply it the metadata associated with analytics services. The FAIR principal addresses who to be findable, be accessible, be interoperable, and be reusable. In Figure \ref{fig:as-fair-do} we explicitly augmented the general FAIR principle with terminology so it can apply to analytics services. The augmentations are colored in red.

\begin{figure}[htb]
\resizebox{1.0\textwidth}{!}{
\tcbox{
\begin{tabular}{p{1cm}p{15cm}}

\multicolumn{2}{l}{To be Findable:} \\
F1 & \textcolor{red}{analytics services metadata} are assigned a globally unique and persistent identifier \\ 
F2 & \textcolor{red}{analytics services} data are described with rich metadata (defined by R1) \\
F3 &  \textcolor{red}{analytics services metadata} clearly and explicitly include the identifier of the data related to the analytics services it describes \\ 
F4 & \textcolor{red}{analytics services metadata} are registered or indexed in a searchable resource \\
\multicolumn{2}{l}{To be Accessible:} \\
A.1 &  \textcolor{red}{analytics services metadata} are retrievable by their identifier using a standardized communications protocol \\
    A1.1 & \textcolor{red}{analytics services} the protocol is open, free, and universally implementable \\
    A1.2 & the \textcolor{red}{analytics services} protocol allows for an authentication and authorization procedure, where necessary \\ 
A.2 & \textcolor{red}{metadata} are accessible, even when the data are no longer available \\
\multicolumn{2}{l}{To be Interoperable:}\\
I1. & \textcolor{red}{analytics services metadata} use a formal, accessible, shared, and broadly applicable language for knowledge representation. \\
I2. &  \textcolor{red}{analytics services metadata} use vocabularies that follow FAIR principles \\
I3. &  \textcolor{red}{analytics services metadata} include qualified references to other metadata \\
\multicolumn{2}{l}{To be Reusable:} \\
R1. & \textcolor{red}{analytics services metadata} are richly described with a plurality of accurate and relevant attributes \\
R1.1 & (meta)data are released with a clear and accessible data usage license \\
R1.2 & (meta)data are associated with detailed provenance \\
R1.3 & (meta)data meet domain-relevant community standards \\
\multicolumn{2}{l}{To be Deployable:}\\
D.1 & \textcolor{red}{analytics services metadata} ... \\
\multicolumn{2}{l}{To be Operational:} \\ 
O.1 & \textcolor{red}{analytics services metadata} ... \\ 
\end{tabular}
}
}
\caption{Fair guiding principles adapted to analytics services: Analytics Services - FAIR - Deployable and Operational
(AS-FAIR-DO).}\label{fig:as-fair-do}
\end{figure}


 




\subsubsection{Support of Plug-In Analytic Services}
\subsubsection{Orchestration of Infrastructure and Services}
\subsubsection{Orchestration of Big Data Applications and Experiments}
\subsubsection{Reusability}
\subsubsection{Execution Workloads}
\subsubsection{Security and Privacy Fabric Requirements}

\subsection{Analytics Service Reusability}
\subsubsection{Packaging}
\subsubsection{Encapsulation}


\subsubsection{Analytics Service Findability}

\begin{figure}
\begin{center}
\begin{forest} for tree={
    grow=east,
    growth parent anchor=east,
    parent anchor=east,
    child anchor=west,
    edge path={\noexpand\path[\forestoption{edge},->, >={latex}] 
         (!u.parent anchor) -- +(5pt,0pt) |- (.child anchor)
         \forestoption{edge label};}
}
[Analytics Services, root
    [Search, onode
        [Catalog, tnode]
        [Registry, tnode]
    [Reusability, onode
        [API, tnode]
    [Compute, onode
        [HPC/batch, tnode]
        [VM, tnode]
        [Desktop, tnode] 
        [Edge, tnode]
        [Container, tnode] ] ]
\end{forest}
\end{center}
\caption{Analytics Services Categories}
\end{figure}

\subsubsection{Analytics Service Catalogue}

\paragraph{Motivation.} Cloud providers offered a considerable set of analytics services to their customers. There are many analytics services available. A user needs to be able to quickly obtain an overview of such available services. This helps identifying further actions in order to evaluate them and identify if further investigation is justified. The catalouge contains enough details to locate the service and evaluate if it is useful. However, it may not provide technical details which are captured by a service registry instead.

\paragraph{Access Requirements.} The catalogue may be public or may be restricted while authorized entities may access it. As analytics services may evolve over time, time dependent versioned descriptions of the services must be able to be included. An organizational entity may manage their own catalogues. It is desirable to have the catalogues be uniform, so that they can be combined into a larger catalogue combining entries of multiple organizations. 

8.2.4.2 and 8.2.4.4 and 8.2.6.2 are all labelled Access Requirements. perhaps we should be more specific

\paragrap{Federation.} The offerings are typically limited to a particular vendor. Users can benefit from a federates service catalogue  to search and explore for needed services by the user. In contrast to a registry a catalogue may not include all technical details but could in contrast include services that lack such details and thus can be the basis of an exploratory process.  A Federated analytics service repository is planned to be hosted on GitHub (LINK TBD)
The catalogue contains the following attributes, many of which are also used in an analytics service registry.

The catalogue is organized as a list of entries, where each entry contains a number of attributes. These attributes may be required or optional. We list in Table \ref{tab:cat}. 


\begin{table}[htb]
\caption{Catalouge attributes}
\label{tab:cat}
\begin{tabular}{p{3cm}p{10cm}p{2cm}}
Name	& Description	& Required \\
\hline
ID	& UUID, globally unique	& \OK \\
Name	& Name of the service	& \OK \\
Title	& Human readable title 	& \OK \\
Public	& True if Public 
(needs use case to delineate what pub private means) & 	\OK \\
Description	& Human readable short Description of the Service	& \OK \\ 
Version	& The version number or tag of the service	& \OK \\
License	& The license description	& \OK \\
Microservice & 	\OK/No/Mixed	& \OK \\
Protocol	& REST	& \OK \\
Owner	& Name of the distributing entity, organization or individual. It could be a vendor.	& \OK \\
Modified	& Modification Timestamp (when first same as created)	\OK \\
Created	& Date on which the entry was first created	& \OK \\
Documentation	& Link to a URL with detailed description of the service	& O \\
Source	& Link to the source code if available	& O \\
Tag(s)	& Human readable common tags that are used to identify the service that are associated with the service	& O \\
Category(s)	& A category that this service belongs to (NLB, Finance, ...)	& O & 
Specification/ Schema	& Pointer to where schema is located &	O \\
Additional metadata	& Pointer to where additional is located including the one here.	& O \\
Endpoint	& The endpoint of the service	& O \\
SLA/Cost	&	& O \\
Authors	& contact details of the people or organization responsible for the service (freeform string)	& O \\
Data	& Description on how data is managed	& O \\
\hline
\end{tabular}
\OK = Required; O = Optional
\end{table}


\subsubsubsection{Analytics Service Registry}

\paragraph{Motivation.} 
The goal of a federated analytics service registry is to establish federated registries to locate and consume analytics services with persistent identifiers across organizations. 

A service registry can serve as a public, private, or federated registry. The first two properties define if the registry is public or private. In case of a private registry proper security measures need to be taken into account to govern access. Our framework does not make any recommendations about the security framework chosen and it is up to the implementer to specify it. In case of a federated registry, more than one registry can be joined, to provide the user the impression of a single registry.

Within the analytics services we distinguish two classes. The first class are instantiated (running) services that are offered by a service provider and allow direct reuse. The second class are library providers that distribute analytics activity not as an instantiated service, but as a source code library which can be deployed as a service. 


[~ use case] A user wants to find an analytics service and needs to identify candidate services based on their descriptions and features. A user wants to find services quickly and therefor expects modern keyword search and taxonomy / faceted search / query functionalities; as well as descriptions that facilitate location and identification of relevant /  appropriate analytics services, from the registry.

The registry contains enough details to not only locate the service, but also how to use it.

\paragraph{Access Requirements.} Public Analytics Service Registry. Public analytics [discovery] services are intended to allow users to find publicly hosted services. The information provided includes the provider, [x], and [y], and / thus reduce users' efforts in locating relevant services.

\begin{description}

\item[Levels of Assurance (LoA) in User Identity] Most readers should be familiar with functionality to “sign in with ORCHID, or Facebook” or something known to the user. In general identity management scenarios, this provision enables what is referred to as “guest identities,” which is useful for many users who are interested in invoking low level activities or less sensitive operations. With respect to federated service authentication and authorization, OIDC guest identities meet a low level of assurance. In contrast, users with higher LoAs are afforded permissions to perform to privileged activities or gain access to more sensitive xyz. 

\item[Multi factor Authentication in User Identity] A means for authenticating users via “two or more types of authentication.” An MFA instrument can elevate a user’s level of assurance profile. RAF and IGTF are examples of such assurance framework standards.  OpenID Connect, SAML, and X.509 are examples of services that expose interfaces for multiple authentication. 

\item[Private Analytics Service Registry. []analytics] Services stored in private registries are only available to authenticated and authorized / member users. Private registries allow providers to build virtual organizations [/ VOMS] that advertise specialized services to its user community. In contrast to a public analytics registry, access controls in private registries are more restricted. In addition, different group privileges may restrict the visible analytics service to the user. See related sub sections on user identity and levels of user priveledge... 

\item[Federated Analytics Service Registry.] A user wants to make selection decisions regarding which service to use. Analytics service brokers / providers therefore offer a federated analytics service in which multiple services from multiple providers are included. Rather than having to visit multiple, separate providers' registries, the user can visit the federated registry of the analytics broker to lookup all potentially suitable services, via a single interface / browser. It may be expected that federated registries abstract the technical effort that casual users would experience during location and inspection of published analytics services. 
Underlying analytics service registry technologies leverage cross - organization persistent identifiers, enhanced with information that the original service provider may not have available, and xyz. such "enrichment" may could include for example, cost comparisons, or [some type of] ratings from its user community.

\item[Enhanced Analytics Service Registry.] Both public and private registries my need to be enhanced by providing detailed information so the user has a better understanding of the offering and allows comparison to similar artefacts maintained / published in the registry. Information details may include for example, benchmark information, service level agreements, or cost measures such as carbon cost, or technical limitations such as storage access and availability for big data.
\end{description}

\paragraph{Registry Namespace.} To allow uniform integration of entries into a unified namespace, URLs are used to distinguish the services. This includes two different entities. Firstly, an entity that defines the code base of a service. Such a code base could be for example hosted on publicly accessible code repositories. Secondly, the namespace could include instantiated analytics service endpoints that define a running instance of an analytics service.                  
Possible tangent to PMML. TODO{russell}{research}

The attributes are listed in Table \ref{tab:reg}. Some attributes may be optional and may be dependent on if they are deployed services, or contain a library that may be deployed.

\paragraph{Describe service provider, library provider infrastructure provider... Move this sub section to its own 'concept' section, or to some other separate section, perhaps the "Landscape."}

{potential text for describe service provider, infra provider: describing federated services and infrastructure providers but likely better for a Federated Access Landscape section:}

Two essential parts of a ‘federation’ infrastructure are: the definition, advertising, and discovery of services; and the secure access to the services [via federated identity]. 
Infrastructure capabilities for ‘unified service discovery’ in federated service registries are not mature. Service / infrastructure providers especially, face many challenges. For example, various registry infrastructures provide support for different user access methods / technologies, but too often are not compatible with the heterogeneous access methods of other [xyz] which desire to interoperate with the federation. 
The following technology review briefly outlines existing projects that have implemented bindings or integrations of service discovery [SD] and federated identity management and user authentication technologies, as they may relate to the federated access.
Sub section 1: 	Notable Implementations and Related Technologies
Sub section 2:	Identity Federation
Sub section 3: 	Terms, Definitions, Acronyms [{gregor} move to appropriate location]

Section 1a: Notable federated access infrastructures 
Type A: 
Infrastructures which provide resources to a broad audience with a wide range of user requirements. Oriented more toward sharing, and potentially supportive of ‘open science;’ but also requiring more attention to AAI. 
XSEDE. The extreme science and engineering discovery environment is an HPC and grid based information system. Supports a number of identity capabilities, and credential translation; and supports several integrations and provides a wide range of resources. 
LHC. The Large Hadron Collider Computing Grid. 
EOSC-Hub, a super federation of the EGI, EUDAT, and INDIGO projects, also serving a broad audience of users. 
EGI and EUDAT. Both the European Data Infrastructure and the European Grid Infrastructure are federated environments / collaborative infrastructures providing functionality for: [AAI requirements [in the form of:]] SSO, delegation, non-web federated access, multiple authentication protocols, LoAs, and distributed authorization; and [SD requirements [in the form of:]] service discovery, unified API, replication, hierarchies, service ingo lifecycle mgmt., and common information model. Both EGI and EUDAT provide a command line interface option for user access. EGI is X.509 based. 
Type B: 
Infrastructures which provide resources of a domain specific nature to smaller communities. 
CLARIN [Common Language Resources and Technology Infrastructure] research infrastructure. A SAML based federation of repositories, supporting SSO for members, and utilizing a registry discovery service [driven by the Language Resource Switchboard [a text analysis application]. Unlike EGI and EUDAT, CLARIN provides a user workflow engine. CLARIN does not provide a command line interface as a user access method; and does not support multiple authentication protocols or LoAs. 
Section 1b: Related technologies / middleware[s], and projects: 
AARC Blueprint architecture [BPA] project. Active in harmonization. 
Argus. An authorization service based on SAML 2.0, VOMS, and X.509 specifications. No GUI. Limited scalability. 
B2ACCESS: a suite of AAI proxy services including identity management, FIM, and credential translation. Deployed by notable research infrastructures including CLARIN, EOSC-Hub, and EPOS.  
BDII. An information service infrastructure service discovery technology deployed by EGI. BDII presents several challenges, in part due to its reliance on LDAP. 
Distributed UDDI Deployment Engine [DUDE] improves on UDDI, a service discovery technology. Limited scalability. 
eduGAIN; and the Research and education identity federations [REFEDS]. eduGAIN is a SAML based federation interconnect / federated AAI, integrated with ELIXIR. REFEDS is an ongoing project to develop attribute harmonization across national federations. 
ELIXIR. A large research infrastructure. Provides federated authentication / AAI technology for web browser style access but does not support non web style access. Supports low LoA social network identities. No support for credential translation. Integrates with eduGAIN. 
European Middleware Initiative [EMI] Registry. An open source client – server federated registry that enables federated authentication, service discovery, and other cross infrastructure services. EMIR relies on JSON, GLUE, and X.509; and integrates with B2ACCESS for AAI. European Plate Observing System [EPOS]. Very large research infrastructure with thousands of users. 
FIWARE. An initiative providing an authorizations API, exposing an HTTP interface, ergo it has no GUI. Does not support delegation of policies. 
GOCDB. An information service infrastructure service registry technology deployed by EGI and EUDAT. 
GLUE [2.0]. Grid Lab Uniform Environment information model. Standards-based / “open standard.” 
IGTF. Interop global trust federation. An infrastructure authentication technology used by CLARIN, EGI, and EUDAT. 
Helmholtz data federation project. Development of an AAI proxy that provides credential translation and interoperability between different providers with incompatible protocols, resulting in federated identity management. 
Metacomputing directory service [MDS]; and ISIS. Peer to peer service discovery technologies. 
Moonshot. Federated authentication / AAI project, offering ‘non web’ browser access. Developed by IETF. Several shortcomings include lack of support for OIDC or X.509.
MWSDI. Groups registries into domains, and groups domains into federations. Limited scalability.
ORCHID. A commercial, author oriented identity service provider with limited integration capability and low levels of assurance. SCOPUS is another author oriented identity service provider. 
PRACE. A large European HPC infrastructure based on PKI AAI. Limited to X.509 services; does not offer OIDC or SAML based identity federation. 
SIRTFI. Security incident response trust framework for federated identify. Part of REFEDS. 
UNICORE. Centralized HPC service registry middleware. The UNICORE FTP is not accessible through web browser interfaces. ckre CIM / CIS.
VOMS. Virtual organization management service. A user authorization technology deployed by EGI. 
WSO2 Identity Server. An authorization service with Active Directory, JDBC, and LDAP integration. Not suitable for distributed management of access control policies. Tightly bound to FIM, which creates challenges for other systems which have established authentication schemes. 
AT&T XACML. Xacml is an RA. NDG XACML
OpenAZ 
Axiomatics policy server [commercial]
The incomplete parts of the listing can be supplemented by russell. 
Research and education federations group assurance frame [RAF]
Section 2: Identity Federation Architectures
Simplistically stated, user identity is the first step in federated authentication, and connection to target infrastructure / service provider endpoints. Generally speaking, there are three types of architectures. 
Type 1: ‘Mesh:’ the most widely adopted model [74], however substantially more complex than the others. 
Type 2: Hub and spoke with distributed login: AARC’s Blueprint Architecture [BPA], and B2ACCESS are adopters of this architecture.  
Type 3: Hub and spoke with central login: this model is not suitable for large scale management of access policies. 

Credential translation. SAML identities are not compatible with OIDC identities. 2nd, service [provider] federations and identity [provider] federations have incompatible attribute dictionaries and authentication protocols. In a sample 107 / 131 EUDAT implementation, the credential translator consists of an OAuth2 authorization server, and a certificate authority and federation database both part of 108 / 132. The user first authenticates with an identity provider, and then may request a token from the translation OAuth2 server, which responds by allowing the user to retrieve a token. The user then asks the translation CA for a credential. The CA then retrieves user attributes from the federation database, and provides the user with a short lived x.509 credential. All of this has taken place in a security environment. The user may now use her certificate to access services, in a service area. 
Section 3: Terms, Definitions, Acronyms [acronyms are on doc 2]
Authentication: a credential. Common authentication refers to the use of a single credential to access multiple infrastructures, resources, services, etc. [1]
Authorization: dealing with “what level of service is allowed to be accessed, rather than who is accessing. [1]
Service discovery: [53]
Web service: “software processes that enable fetching, adding, editing, or deleting data.” [2]

Additional concepts to be added to this section – 
LoA calculation
[memon 99]. Attribute provenance
[memon 71]. Metadata schema
[memon 27 / 51] guest identity
}
todo: Lightening strike outage, server outage.

\begin{table}[H]
\caption{Registry attributes}\label{tab:reg}
\resizebox{\columnwidth}{!}{%
\begin{tabular}{p{3cm}p{11cm}p{0.5cm}p{0.5cm}}
Name	& Description	& \rot{\shortstack{Service\\ provider}}	& \rot{\shortstack{Library\\ provider}} \\
\hline
ID & 	UUID, globally unique &	\OK &	\OK \\
Name & 	Name of the service	& \OK	& \OK \\ 
Title & 	Human readable title &	\OK	& \OK \\
Public	& True if Public
(needs use case to delineate what pub private means) & 	\OK & \OK \\
Description	& Human readable short Description of the Service	& \OK & 	\OK \\
Endpoint &	The endpoint of the service	& \OK	&  N/A \\
List of Input Parameters &
	A list of parameters to the service. The parameters have each the form of name, function, type, value, access. The type indicates the data type. The access indicates if the parameter is a data stream, database, single value/function, event.
The function responds to a different function in case multiple are provided by the service.	& \OK	& \OK \\ 
List of Output Parameters 
  style (event, stream, data)
  value
  timestamp & 
	List of responses cast by the service. The responses have the form of function, name, type, value, access, timestamp. The type indicates the data type. The access indicates if the parameter is a data stream, database, single value/function, event.
The function responds to a different function in case multiple are provided by the service. & 	\OK 	& \OK \\
Version	& The version number or tag of the service	& \OK	& \OK \\
License	& The license description	& \OK	& \OK \\
Protocol & 	REST	& \OK	& \OK \\
Modified & 	Modification Timestamp	& \OK& \OK \\
Owner	& Name of the distributing entity, organization or individual. It could be a vendor.	& \OK	& O \\
Author &	Contact details of the people or organization responsible for the service	& O	& \OK \\
Tags &	Human readable common tags that are used to identify the service that are associated with the service	& O & O \\
Categories &	A category that this service belongs to (NLB, Finance, ...)	& O & O \\
Created	& date and time on which the analytics service was instantiated or created	instantiated	& \OK & \OK \\
Heartbeat &	State and timestamp of the last check when the service was active	& O & 	N/A \\
Documentation &	Link to a URL with detailed description of the service
Source	Link to the source code if available	& O & O \\
Specification &	Pointer to where specification schema is located	& O &  O \\
AdditionalMetadata	& Pointer to where additional is located including the one here.	& O &	O \\
SLA	& Serves level agreement including cost	& O 	& O \\
CachingInterval	&If a service is accessed a lot, the caching interval can be used to put a limitation on the Response with an LRU cache	& O &	N/A \\
DataIntegration &	In case of big data the data cannot be provided as a parameter to the analysis function. Instead, we need to provide the data as endpoint. However, often tata may need to be uploaded or can be downloaded. In this case this field provides the upload and download endpoints and the protocol to access the data	& O &	O \\
\hline
\end{tabular}
}
\OK = Required; O = Optional
\end{table}

\subsubsection{Benefits of a federated analytics service registry}

A service registry can publicize and improve end user access to data from different sources, by overcoming some of the challenges inherent in describing and surfacing document content and format. Publication, and discovery of information resources are enriched with metadata enabling the findability and reusability of a service supporting the FAIR principle. While describing the interfaces and allow for the instantiation or the reuse of already instantiated services we address the accessibility and interoperability. With respect to analytics as a service, end users should be able to find various analytic services and similar services without having to individually search multiple ‘locations’ or databases, each built to operate on its own, unique storage and retrieval constructs. Through these descriptions automated service integration can be provisioned while targeting not only the functionality involved, but also allowing service level considerations to be addressed. Furthermore, such analytics services could provide significant security implications such as the protection of a database while only exposing a subsect of approved analytics functions that are executed on the data sets. This includes partial and controlled sharing of data mashup that can be made available to the community and registered to make reuse easier without everyone having to replicate the service.




\subsubsection{Analytics Service Pipelines}

\paragraph{Motivation.} In many cases a big data analysis is split up in multiple subtasks. These subtasks may be reusable in other analytics pipelines. Hence it is desirable to be able to specify and use them in a coordinated fashion allowing reuse of the logic represented by the analysis. Users must have a clear understanding on what the analysis is doing and how it can be invoked and integrated. 

\paragraph{Access Requirements.} The analysis must include a clear and easy to understand specification that encourages reuse and provides sufficient details about its functionality, data dependency and performance. Analytics services may have authentication, autorotation and access controls build in that enable access buy users controlled by the service providers.

\subsubsection{Federated Analytics Service Catalogue}
\subsubsection{Catalogue Attributes}
\subsubsection{Federated analytics service Registries}
\subsubsection{Registry Attributes}

\subsection{Resource Accessibility}
\subsubsection{Resource Management}
\subsubsection{Security}




\section*{References}
\addcontentsline{toc}{section}{References}
\bibliographystyle{format/techpubs}
\bibliography{References}

\appendix

\section{Acronyms and Terms}


\begin{quote}
    \input{dictionary}
\end{quote}



\section{Implementation Examples}

This section contains a number of implementation examples that follow the document specifications.

\subsection{Analytics Catalog}
\addcontentsline{toc}{subsection}{Appendix B.1: Anlaytics Catalog}

TBD

\subsection{Analytics Registry}

TBD

\section{Change Log}
\addcontentsline{toc}{section}{Appendix B: Change Log}

If updating document with errata, detail changes made to document.
The updates are recorded in reverse chronological order with the most recent change first.\\


\begin{tabular}{ p{3cm}p{1cm}p{9cm}}
Date            & Initials & Comment \\
\hline
Januarry 2022   & Gregor & merging dictionary terms into the list of terms. \\
December 2021   & RcR 	& \TODO{russel}{Gregor: On the organization of the outline sections and sub sections: 7.2 In my domain, the concept behind the term 'Landscape' would not be related to Specification. I am interested to work on this sub section so I amm looking forward to some clarification on how you see Landscape as a sub of Specification. Depending on the 'vision' of the topics / content for 7.2 and Appendix B, (which I am also interested to work on) they may have some overlap.} \\
September 2021  & GvL & Spelling corrections \\
August 2021     & Gvl & 	Add dictionary \\
                & GvL & Add executive summary \\
                & GvL & Fix introduction \\
                & GvL & Add catalogue \\
                & GvL & Add registry \\
July 2021       & GvL & Integrated notes on Metadata by Russell \\
March 20, 2021  & GvL & First Draft \\
\hline
\end{tabular}


\end{document}
