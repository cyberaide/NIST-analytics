\section{Security analysis}

\TODO{ add the section }


This use case has the implict requirements needing the following
aspects to be addressed by the framework we develop.

\begin{enumerate}

\item{\bf AS vendor neutral cloud and computer service integration.} ...

  \begin{enumerate}
  \item {\bf AS in cloud.} ...
  \item {\bf AS in LCCF.} ...
  \item {\bf AS in microservices.} ...
  \end{enumerate}

\item{\bf AS architecture.} ...

  \begin{enumerate}
  \item{\bf AS vendor neutral interfaces.} ...
  \item{\bf AS REST.} ...
  \item{\bf AS layers such as interface, service layer, and provider layer.} ...
  \end{enumerate}

\item{\bf AS workflow.} ...

  \begin{enumerate}
  \item{\bf AS catalog and registry.} ...
  \item{\bf AS cooperation.} ...
  \item{\bf competition.} ...
  \item{\bf AS orchestrator.} ...
  \end{enumerate}


\item{\bf AS calculation.}

  \begin{enumerate}
  \item{\bf AS with DL.} ...
  \item{\bf AS data analytics.} ...
  \end{enumerate}

\item{\bf AS security.} ...

\end{enumerate}
The legacy systems; Electronic Health Records (EHR), Medical Imaging (MI), and Genomics are healthcare data sources that process to drive clinical and operational on the individual patient and population levels. However, in a world where data is growing at unprecedented rates a new strategy and a new set of tools are required to capture the promises of Big Data Analytics in the healthcare space. Multi Hybrid Cloud, and edge computing rapidly changing how healthcare data is handled and creating novel opportunities to leverage large-scale data aggregation to train Analytics Services (AS) models, that enable operational efficiency and diagnostic assistance tools (with required selected data set) in real-time on each care provider unit (Emergency (ER), Critical Care Unit (CCU), and others).
Data Analytics services are undisputedly the technology of the future for the healthcare industry to develop complex algorithms that can learn and, resolve the behavior in the discovery or diagnosing of the patient care. In doing so, the health organizations have helped prove that data analytics and ML work.
From big legacy data to good big data:
The legacy big data generated enormous limited potential public health data with antiquated capabilities and no core design to support accepted standards of medical ethics.
To efficiently manage to explode the big data tagging, and provide health diagnoses for patient care in real-time to diagnosis and even prediction of disease outcomes for patients.
To enable population-scale data harmonization for research and to reduce costs associated with data transfers,
To assure alignment with accepted standards of medical ethics,
To guarantee data integrity,
To assure seamless and secure access for system interoperability and data sharing, and (i.e. https://www.weforum.org/agenda/2021/11/healthcare-cybersecurity/)


