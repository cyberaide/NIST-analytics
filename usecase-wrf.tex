\subsection{Case Study Title: Numeric Weather Prediction}

Contributors: Wo Chang, Daniel Keirouz (former summer intern), NIST, US

\subsubsection{Background}

Large amounts of weather data are produced continually and stored in many different databases.
Accurate weather predictions require large amounts of processing power to accurately simulate
conditions worldwide at a high resolution and frequent intervals. One of the most computationally
consuming parts of a reliable weather model is the microphysics scheme. The current microphysics
scheme, Weather Research and Forecasting (WRF) Single Moment 6-class Microphysics (WSM6),
simulates the processes in the atmosphere that lead to the formation and precipitation of rain, snow,
and graupel and requires complex floating-point operations needing to be performed on vast
amounts of data for accurate simulations. As computer performance improves, so does the Numerical
Weather Prediction (NWP) models' resolution and accuracy. However, there is still much progress to
be made, as simulation accuracy still falls off significantly for predictions more than 36 hours in
advance. Figure 1 shows the general WRF modeling system flow chat. 

Functionalities and Activities (based on Big Data Application Provider of NBDIF Ref. Architecture)
In this case study, we only focus on two main functionalities, namely WPS and WRF, and their activities.

Figure 2 shows the cross-functional diagram for their actions.

WPS Activities:

\begin{enumerate}
\item geogrid – defines simulation domains and interpolate various terrestrial data sets to the
model grids. Input data available at [1].
\item ungrib – extracts needed meteorological data and packs it into an intermediate file format.
Input data available at [2]
\item medgrid – prepares horizontally interpolate the meteorological data onto the model domain.
Input data from the output of geogrid and ungrib.
\end{enumerate}

WRF Activities:

\begin{enumerate}
\item real – prepares vertically interpolates the output from metgrid, and creates a boundary and initial
condition files with some consistency checks.
\item wrf – generates a model forecast.
\end{enumerate}

Datasets:
[1] \url{http://www2.mmm.ucar.edu/wrf/users/download/get_sources_wps_geog.html}
[2] \url{https://rda.ucar.edu/datasets/ds083.2/#!access}